\documentclass[11pt]{article}
\usepackage{physics}
\usepackage{comment}
\usepackage{amsmath}
\usepackage{float}
\usepackage{tikz}
\usepackage{graphicx}
\usepackage{caption}
\usepackage{subcaption}

\usetikzlibrary{quantikz}

\title{Quantum project: Report 3}
\author{}
\begin{document}
\maketitle

\section{Phase covariant cloning machine}
The circuit for the phase covariant quantum cloning machine (PCQCM) proposed in  \cite{EquatorialQCM} is similar to one for the universal quantum cloning machine (UQCM):

\begin{quantikz}
    \lstick{$\ket{\psi}_{a_0}$} & \qw                  & \qw        & \qw                   & \qw       & \qw                   & \ctrl{2}  & \ctrl{1}      & \targ{}        & \targ{}       &\qw\\
    \lstick{$\ket{0}_{a_1}$}    & \qw                  & \targ{}    & \gate{R_y(\theta_2)}  & \ctrl{1}  & \qw                   & \qw       & \targ{}       & \qw           & \ctrl{-1}      & \meter{}         \\
    \lstick{$\ket{0}_{a_2}$}    & \gate{R_y(\theta_1)} & \ctrl{-1}  & \qw                   & \targ{}   & \gate{R_y(\theta_3)}  & \targ{}   & \qw           & \ctrl{-2}     & \qw            & \meter{}   \\
\end{quantikz}

Note that in this circuit, the copies are made in the two ancillae (and not in the same Hilbert space as the input qubit).

The only significant difference is the preparation of the two ancillae. In this case, different angles of rotation are used.
It turns out that in order to optimally clone the states on the $x-z$ equator, we have to choose the following angles:
\[
    \theta_1=\theta_3=2\arcsin(\sqrt{\frac{1}{2}-\frac{1}{2\sqrt{2}}})=\frac{\pi}{4}=\theta, \quad \theta_2=0.
\]

Hence, we can rewrite the circuit above removing the rotation $R_y(\theta_2)$:

\begin{quantikz}
    \lstick{$\ket{\psi}_{a_0}$} & \qw                  & \qw         & \qw       & \qw                   & \ctrl{2}  & \ctrl{1}      & \targ{}        & \targ{}       &\qw\\
    \lstick{$\ket{0}_{a_1}$}    & \qw                  & \targ{}     & \ctrl{1}  & \qw                   & \qw       & \targ{}       & \qw           & \ctrl{-1}      & \meter{}         \\
    \lstick{$\ket{0}_{a_2}$}    & \gate{R_y(\theta_1)} & \ctrl{-1}   & \targ{}   & \gate{R_y(\theta_3)}  & \targ{}   & \qw           & \ctrl{-2}     & \qw            & \meter{}   \\
\end{quantikz}

The first two CNOT gates in this circuit act as a SWAP gate (because we have $\ket{0}_{a_1}$ as an input state). 
It is now clear that these two CNOT gates are not necessary. We can delete them, provided that we move $R_y(\theta_3)$ to the second qubit. Since $\theta_1=\theta_3=\theta=\frac{\pi}{4}$, we obtain the following circuit:

\begin{quantikz}
    \lstick{$\ket{\psi}_{a_0}$} & \qw                  & \ctrl{2}  & \ctrl{1}      & \targ{}        & \targ{}       &\qw\\
    \lstick{$\ket{0}_{a_1}$}    & \gate{R_y(\pi/4)} & \qw       & \targ{}       & \qw           & \ctrl{-1}      & \meter{}         \\
    \lstick{$\ket{0}_{a_2}$}    & \gate{R_y(\pi/4)} & \targ{}   & \qw           & \ctrl{-2}     & \qw            & \meter{}   \\
\end{quantikz}

We have managed to greatly simplify the circuit. We have removed 5 CNOT gates from the circuit that we were using on real quantum processors with linear connectivity: the 2 CNOT gates mentioned here and the 3 CNOT gates that were necessary in order to implement the SWAP gate that was necessary to respect the connectivity constraints.
With this new circuit, only the top qubit has to be able to connect to the other two qubits.

Furthermore, the preparation of the input qubit and the measurement of the copies is also simplified. 
Since we are only interested in the $x-z$ equator of the Bloch sphere, we always have either $\phi=0$ or $\phi=\pi$. 
If $\phi=0$ we do not need to perform any rotation around the $z$-axis. If $\phi=\pi$, we can apply a $Z$ gate instead of performing a rotation around the $z$-axis.

As a final observation, since this circuit is much more simple than the one for the universal QCM, the average fidelity over the whole Bloch sphere could be greater in this case than with the universal QCM.

On the equator, the expected fidelity for both the copies is:
\[
    F_{ideal}=\frac{1}{2}+\frac{1}{2\sqrt{2}}\approx 0.854\dots
\]
\subsection{Full Bloch sphere}
Running the circuit on IBM simulator, we got the following fidelities for the two copies:
\[
    F_1=0.82 \pm 0.03, \quad F_2=0.82 \pm 0.03.
\]
\begin{figure}[H]
    \centering
            \includegraphics[totalheight=6cm]{Figures/topviewstarmon5qubit3.png}
        \label{fig:topviewstarmon}
        \caption{Phase covariant QCM run on Starmon. Here it is shown the top view of the Bloch sphere, where it is plotted the fidelity of the second copy (on the physical qubit 3). }
\end{figure}

This is the average fidelity and standard deviation expected when using the phase covariant cloning machine on the full Bloch sphere. 
\begin{table}[H]
    \centering
    \begin{tabular}{|c|c|c|c|c|c|c|c|c|}
    \hline
    \textbf{} & \textbf{Starmon 5} & \textbf{Athens} & \textbf{Ourense} & \textbf{Santiago} & \textbf{Valencia} & \textbf{Vigo} & \textbf{Yorktown} \\ \hline
    $F_1$              & 0.75 & 0.78 & 0.79 & 0.75 & 0.4 & 0.78 & 0.76 \\ \hline
    $\sigma_{F_1}$     & 0.04 & 0.04 & 0.03 & 0.04 & 0.2 & 0.03 & 0.03 \\ \hline
    $F_2$              & 0.76 & 0.82 & 0.78 & 0.73 & 0.5 & 0.79 & 0.81 \\ \hline
    $\sigma_{F_2}$     & 0.04 & 0.03 & 0.04 & 0.04 & 0.2 & 0.04 & 0.01 \\ \hline
    $F_2/F_1$          & 1.01 & 1.05 & 0.99 & 0.98 & 1 & 1.02 & 1.07 \\ \hline
    $\sigma_{F_2/F_1}$ & 0.05 & 0.01 & 0.02 & 0.04 & 1 & 0.03 & 0.04 \\ \hline
    \end{tabular}
\end{table}

Here are also the results found last time with the universal QCM, for comparison.
\begin{table}[H]
    \centering
    \begin{tabular}{|c|c|c|c|c|c|c|c|}
    \hline
    \textbf{} & \textbf{Starmon 5} & \textbf{Athens} & \textbf{Ourense} & \textbf{Santiago} & \textbf{Valencia} & \textbf{Vigo} & \textbf{Yorktown} \\ \hline
    $F_1$              & 0.73 & 0.77 & 0.78 & 0.73 & 0.6 & 0.76 & 0.76 \\ \hline
    $\sigma_{F_1}$     & 0.03 & 0.02 & 0.03 & 0.04 & 0.2 & 0.03 & 0.02 \\ \hline
    $F_2$              & 0.73 & 0.78 & 0.72 & 0.72 & 0.6 & 0.75 & 0.76 \\ \hline
    $\sigma_{F_2}$     & 0.05 & 0.02 & 0.03 & 0.04 & 0.2 & 0.03 & 0.02 \\ \hline
    $F_2/F_1$          & 1.00 & 1.01 & 0.93 & 0.98 & 1.1 & 0.99 & 1.00 \\ \hline
    $\sigma_{F_2/F_1}$ & 0.04 & 0.03 & 0.03 & 0.03 & 0.7 & 0.02 & 0.03 \\ \hline
    \end{tabular}
\end{table}

The phase covariant QCM performs better in all the cases.
Note, however, that for some backends different physical qubits may have been used (for the phase covariant QCM, we let the backend decide the best physical qubits).
Moreover, the performances of the backends change day by day, depending for example on the calibration of the processor.

The result of IBM Valencia is not consistent with the expectations. I think they might have problems with the rotations along the $y$ axis of one of the central qubits.

NB: Starmon went offline before the whole Bloch sphere was covered. 
The last points at the bottom of the sphere that were calculated when it came back online where worse than the previous ones (the difference before and after the stop is clearly visible).
We ran the circuit again obtaining:
\[
    F_1=0.69 \pm 0.07, \quad F_2=0.73 \pm 0.05
\]


\subsection{Only $xz$ equator}

\begin{table}[H]
    \centering
    \begin{tabular}{|c|c|c|c|c|c|c|c|c|}
    \hline
    \textbf{} & \textbf{Starmon 5} & \textbf{Athens} & \textbf{Ourense} & \textbf{Santiago} & \textbf{Valencia} & \textbf{Vigo} & \textbf{Yorktown} \\ \hline
    $F_1$              & 0.79 & 0.81 & 0.84 & 0.78 & 0.5 & 0.80 & 0.76 \\ \hline
    $\sigma_{F_1}$     & 0.03 & 0.02 & 0.01 & 0.02 & 0.2 & 0.02 & 0.04 \\ \hline
    $F_2$              & 0.79 & 0.86 & 0.82 & 0.84 & 0.5 & 0.84 & 0.81 \\ \hline
    $\sigma_{F_2}$     & 0.04 & 0.02 & 0.01 & 0.01 & 0.2 & 0.02 & 0.01 \\ \hline
    $F_2/F_1$          & 0.99 & 1.06 & 0.98 & 1.07 & 1 & 1.05 & 1.06 \\ \hline
    $\sigma_{F_2/F_1}$ & 0.04 & 0.01 & 0.01 & 0.02 & 1 & 0.02 & 0.05 \\ \hline
    \end{tabular}
\end{table}
The quality of the copies on the equator is of course better than the avereage on the Bloch sphere. The average fidelity is close to the theoretical bound.
For Athens, the average fidelity of the second copy is $F_2=0.86$, which is greater than the expected fidelity.
This shouldn't worry too much: first of all there is some statistical uncertainty, moreover the bound is calculated considering a unitary evolution, but when using a real machine, the evolution is not unitary because of noise.
Therefore, the bound does not hold strictly speaking (of course it would hold in absence of noise).

\subsection{Further analysis on Starmon}
We investigated furtherly the phase covariant QCM run on Starmon ($xz$ equator only), namely considering readout correction and different calibrations (the new one is 12/17/2020 - 2:32:41 PM):
).

\begin{table}[H]
    \centering
    \begin{tabular}{|c|c|c|c|c|c|}
    \hline
    \textbf{} & \textbf{Qubits13} & \textbf{Qubits13 RO} & \textbf{Qubits13 new} & \textbf{Qubits13 RO new} \\ \hline
    $F_1$              & 0.79 & 0.80 & 0.78 & 0.80 \\ \hline
    $\sigma_{F_1}$     & 0.03 & 0.03 & 0.03 & 0.03 \\ \hline
    $F_2$              & 0.79 & 0.79 & 0.79 & 0.79 \\ \hline
    $\sigma_{F_2}$     & 0.04 & 0.04 & 0.04 & 0.04 \\ \hline
    $F_2/F_1$          & 0.99 & 0.99 & 1.01 & 0.99 \\ \hline
    $\sigma_{F_2/F_1}$ & 0.04 & 0.05 & 0.05 & 0.05 \\ \hline
    \end{tabular}
\end{table}
The readout correction does not change the results significantly.

\begin{figure}[H]
    \centering
            \includegraphics[totalheight=6cm]{Figures/starmon13qubit1oldcalib.png}
        \label{fig:oldcalib1}
        \caption{Phase covariant QCM run on Starmon (old calibration). Copy on qubit 1.}
\end{figure}
\begin{figure}[H]
    \centering
            \includegraphics[totalheight=6cm]{Figures/starmon13qubit3oldcalib.png}
        \label{fig:oldcalib2}
        \caption{Phase covariant QCM run on Starmon (old calibration). Copy on qubit 3.}
\end{figure}
\bibliography{biblio}
\bibliographystyle{ieeetr}
\end{document}
