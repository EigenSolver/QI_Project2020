\documentclass[11pt]{article}

\usepackage{physics}
\usepackage{comment}
\usepackage{amsmath}
\usepackage{amsfonts}
\usepackage{dsfont}
\usepackage{tikz}
\usepackage{csvsimple}
\usetikzlibrary{quantikz}

\title{Quantum project: Report 3}
\author{}


\begin{document}
\maketitle




\section{Simulation of QKD Attack by a eavesdrpper equipped with UQCM}
In this section we investigate whether we can attack QKD protocol equipped with a UQCM.
\
We briefly review the BB84 QKD protocol\cite{bennett2020quantum} 
\begin{enumerate}
    \item Bob randomly picks a basis $B_i\in \{X,Z\}$ and randomly picks a private bistring $b_i\in \{0,1\}$.
    \item Bob prepare the qubit in state in $\ket{0}, \text{or} \; \ket{1}$ or $\ket{+}, \text{or} \; \ket{-}$ depends on what basis and key he got.
    \item Bob sends his qubit to Alice.
    \item Alice measure the qubit in a random picked basis $\tilde{B}_i\in \{X,Z\}$.
    \item The above process is repeated for several times to send a multiqubit key.
    \item Bob and Alice publish their measurement basis.
    \item Compare the basis and drop the measurement result with different basis.
    \item Sampling the remained measurement, check the result(in public channel). If the error rate is above threshold, abort the distribution process (for possible eavesdrping). 
    \item Preserve all remained measurement and take the result as a key.
\end{enumerate}

\subsection{Intercept and resend Attack} 
Suppose Eve eavesdrping by intercepting and randomly measuring the qubit, then resending a new one based on his measurement result.
The probability of Eve being detected for a single qubit is $\epsilon=1/4$. 

If the remained bitstring is of length $k$, Bob and Alice will preserve $k/2$ bits for error check,
 then probability of a successful eavesdroping without being noticed is

\[
    P(k)=(1-\epsilon)^{k/2}
\]
For a lone key with more than $10^2$ bits, a brief estimation under ideal condition is given by 
\[
    P\approx 10^{-12}
\]
theoretically it's almost impossible to eveasdroping without being detected.

\subsection{Attack with Quantum Copying Machine and Quantum Memory} 
Assuming Eve has a quantum copying machine and a quantum memory capable to store all the qubits
intercepted, then the eavesdroper can partially clone and store all the qubits, 
then measure them after Alice and Bob publish their measurement basis.

The maximum fidelity of UQCM is $F=5/6$. Thus the minimum error rate of a single qubit is $\epsilon=1/6$.

The success rate of Eve still decays exponentially with key length, but with a lower rate.
For a lone key with more than $10^2$ bits, apply the same formula we get
\[
    P\approx 10^{-9}
\]

\section{Error Rate in Noisy Reality}


However, the near-term quantum device is always noisy, which means error is possible when Bob and Alice compare their qubits.
That's why we need a threshold $m$ to allow some extent of error tolerance,
\[
    P(k,p)=\sum_{j=0}^{m} \binom{k/2}{m} \epsilon^j(1-\epsilon)^{k/2-j}
\]

The threshold $p$ should depends on the fidelity of the quantum channel in which Alice and 
Bob transfer their qubits.

If Alice and Bob use a random sampling check to avoid eavesdrppor, suppose the sampling size is $N$, then the average error rate is defined as  

\[
\delta=\frac{|\{x_i|x_i^A \neq x_i^B\}|}{|\{x_i\}|}    
\]


We simulate a single-shot QKD process on Starmon-5, with 64, 128 and 144 key length.
A example table is shown as

\begin{center}
    \begin{tabular}{|l|l|c|c|c|}%
        \hline 
        \bfseries No. & \bfseries Key & \bfseries Measure Basis & \bfseries Bob & \bfseries Eve% specify table head    
        \csvreader[head to column names,
        filter test=\ifnumless{\index}{30}
        ]{../QKD/QKD_Simulation_128.csv}{}% use head of csv as column names
        {\\
        \hline\index & \key & \bases & \Bob & \Eve}% specify your coloumns here
    \end{tabular}
\end{center}


% Consider a channel single-bit depolarizing noise characterized by $p$
%     \[
%         \mathcal{D}_p: \, \rho\rightarrow p\rho+(1-p)\mathds{1}/2
%     \]
% Then the probability of get a bit filp for single qubit is $\epsilon'=(1-p)/2$.

% Then we can derive the flip probability of a single bit
% \[
% \epsilon''=(1-\epsilon)\frac{1-p}{2}+   \epsilon\frac{1+p}{2}=\frac{1}{2} (1-p+2p\epsilon)
% \]

% When the net error reach the same level with the environment noise,
% we can say that error introduced by Eve can't be distinguished.
Where the measurements with different basis are ignored, the measurement basis here is the 
common basis for Alice and Bob. 

The average error rate only relys on the noise of our quantum hardware, 
it's not related with key length or sampling size.

The average result we obtained is 
\[
\delta=(0.67774\pm 0.02140)    
\]
\begin{figure}
    \centering
    \includegraphics[]{qkd145_error.png}
    \caption{Sampling Error for 144bit QKD}
\end{figure}
\end{document}

