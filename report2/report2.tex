\documentclass[11pt]{article}
\usepackage{physics}
\usepackage{comment}
\usepackage{amsmath}
\usepackage{float}
\usepackage{tikz}
\usetikzlibrary{quantikz}
\title{Quantum project: Report 1}
\author{}
\begin{document}
\maketitle
\begin{comment}
\section{Introduction}
Copying arbitrary quantum information is forbiden by the laws of quantum mechanics. Quantum communication protocols, e.g. QKD, rely on the fact that an eveasdroper will be noticed when intercepting and reading the information, which ensures that communication remains private. In this work we explore an imperfect quantum copy machine and its implementation on quantum computting platforms such as Quantum Inspire (QI). First, we will analyze the circuit that implements this copy machine and obtain a clear idea of what the output state is. Secondly, we implement this circuit for all single qubit states and show the results of the devices Starmon5 and the QXsimulator.
\section{Universal quantum copy machine}

The quantum circuit corresponding to the UQCM contains two stages, as can be seen in the following circuit,\\
\begin{center}
\begin{quantikz}\label{circuit:full}
\lstick{$|\psi \rangle_{a_0}$}   &\qw &  \gate[3][2cm]{U_{copy}}&\qw\\
\lstick{$|0\rangle_{a_1}$} & \gate[2][2cm]{U_{prepare}}  && \qw\\
\lstick{$|0\rangle_{b_1}$} &   & &\qw
\end{quantikz}.
\end{center}


\subsection{Preparation of the state}
The first stage of the UQCM starts before interacting with the input qubit. The two bottom qubits are required to be in an arbitrary state,
\begin{equation}
| \phi \rangle_{a_1b_1} = C_1|00\rangle + C_2|01\rangle + C_3|10\rangle +C_4 |11\rangle.
\end{equation}
For such purpose we will use the following circuit with an input state $\ket{00}_{a_1b_1}$. 
\begin{center}


\begin{quantikz}\label{circuit:full}
\lstick{$|0\rangle_{a_1}$} & \gate[2][1cm]{U_{prepare}}  & \qw\\
\lstick{$|0\rangle_{b_1}$} &   &\qw
\end{quantikz}=

\begin{quantikz}[slice all]\label{circuit:full}
\lstick{$|0\rangle_{a_1}$} & \gate{R_y(\theta_1)}  & \ctrl{1}& \qw  &\targ{} &\gate{R_y(\theta_3)}&\qw\\
\lstick{$|0\rangle_{b_1}$} & \qw & \targ{}& \gate{R_y(\theta_2)}  &\ctrl{-1} &\qw&\qw
\end{quantikz}.


\end{center}
We analyse each stage of the preparation process. Simply denote $\ket{00}_{a_1b_1}$ as $\ket{00}$, where the 
qubit from left to right side is always $a_1b_1$.

The rotation gate here is defined by 
\begin{align*}
    R_y(\theta)&
    =\cos(\theta/2) \hat I-i \sin(\theta/2)\hat Y
    =\begin{pmatrix}
        \cos(\theta/2) & -\sin(\theta/2)\\
        \sin(\theta/2) & \cos (\theta/2)
    \end{pmatrix}
\end{align*}

And the formula can be given by,
\begin{align*}
|\phi_0\rangle &= |00 \rangle \\
|\phi_1\rangle &= (\cos(\theta_1/2)|0\rangle + \sin(\theta_1/2)|1\rangle)|0\rangle\\
|\phi_2\rangle & = \cos(\theta_1/2)|00\rangle + \sin(\theta_1/2)|11\rangle\\
|\phi_3\rangle &=\cos(\theta_1/2)|0\rangle(\cos(\theta_2/2)|0\rangle + \sin(\theta_2/2)|1\rangle) + \sin(\theta_1/2)|1\rangle(-\sin(\theta_2/2)|0\rangle + \cos(\theta_2/2)|1\rangle)\\
&=\cos(\theta_1/2)\cos(\theta_2/2)\ket{00}+\cos(\theta_1/2)\sin(\theta_2/2)\ket{01}-\sin(\theta_1/2)\sin(\theta_2/2)\ket{10}+\sin(\theta_1/2)\cos(\theta_2/2)\ket{11}\\
|\phi_4\rangle &=\cos(\theta_1/2)\cos(\theta_2/2)|00\rangle + \cos(\theta_1/2)\sin(\theta_2/2)|11\rangle- \sin(\theta_1/2)\sin(\theta_2/2)|10\rangle + \sin(\theta_1/2)\cos(\theta_2/2)|01\rangle\\
|\phi_5\rangle&=(\cos(\theta_3/2)|0\rangle + \sin(\theta_3/2)|1\rangle) 
(\cos(\theta_1/2)\cos(\theta_2/2)|0\rangle+\sin(\theta_1/2)\cos(\theta_2/2)|1\rangle)\\
&+(-\sin(\theta_3/2)|0\rangle + \cos(\theta_3/2)|1\rangle)
(\cos(\theta_1/2)\sin(\theta_2/2)|1\rangle- \sin(\theta_1/2)\sin(\theta_2/2)|0\rangle)
\end{align*}
Finally, we found that the coefficients of the final state depends on the rotation angles,

\[
    |\phi\rangle= C_1|00\rangle + C_2|01\rangle + C_3|10\rangle +C_4 |11\rangle.
\]
From which we can observe that,

\begin{align}
    C_1 &=     \sin(\theta_1/2)  \sin(\theta_2/2)  \sin(\theta_3/2) +  \cos(\theta_1/2) \cos(\theta_2/2)  \cos(\theta_3/2)\\
    C_2 &=   \sin(\theta_1/2)  \cos(\theta_2/2) \cos(\theta_3/2) -   \cos(\theta_1/2) \sin(\theta_2/2)  \sin(\theta_3/2)   \\
    C_3 &=   \cos(\theta_1/2)  \cos(\theta_2/2) \sin(\theta_3/2) -  \sin(\theta_1/2)  \sin(\theta_2/2)  \cos(\theta_3/2)\\
    C_4 &=   \sin(\theta_1/2)  \cos(\theta_2/2) \sin(\theta_3/2)  +    \cos(\theta_1/2) \sin(\theta_2/2) \cos(\theta_3/2)
\end{align}

In order to obtain the desiered preparation state described in the following section, the following angles were used:
\[
    \cos(\theta_1)=\frac{1}{\sqrt{5}}, \quad \cos(\theta_2)=\frac{\sqrt{5}}{3}, \quad \cos(\theta_3)=\frac{2}{\sqrt{5}}
\]

\subsection{Copying proccess}
Once we have prepared the state, we do the copy process, which can be described as controlled entanglement between the input and prepared qubits. The circuit corresponding to the copy process can be observed in the following circuit, where the prepared initial state and the input state that we consider are given respectively by,
\begin{equation}
|\phi\rangle^{(prep)}_{a_1,b_1} = \frac{1}{\sqrt{6}} (2|00\rangle + |01\rangle + |11\rangle),\qquad |\psi\rangle^{(in)}_{a_0} = \alpha|0\rangle +\beta|1\rangle.
\end{equation}
\begin{center}
\begin{quantikz}\label{circuit:full}
% \lstick{$|\psi\rangle_{a_0}$}   & \gate[3][2cm]{U_{copy}}&\qw\\
% \lstick{$|0\rangle_{a_1}$} & & \qw\\
% \lstick{$|0\rangle_{b_1}$} &  &\qw
\lstick{}   & \gate[3][2cm]{U_{copy}}&\qw\\
\lstick{} & & \qw\\
\lstick{} &  &\qw
\end{quantikz}=\begin{quantikz}[slice all]\label{circuit:full}
\lstick{}   & \ctrl{1}&\ctrl{2}&\targ{}&\targ{}&\qw\\
\lstick{} &\targ{} & \qw & \ctrl{-1} &\qw&\qw\\
\lstick{} & \qw&\targ{}&\qw&\ctrl{-2}&\qw
\end{quantikz}.
\end{center}
We consider that the input state of the copy machine is $|\Psi_0\rangle = |\psi\rangle^{(in)}_{a_0}|\phi\rangle^{(prep)}_{a_1,b_1}$. Then, it will transform in each stage of the copy process as,
\begin{align}
|\Psi_1\rangle &= \frac{\alpha}{\sqrt{6}}|0\rangle(2|00\rangle +|01\rangle + |11\rangle) + \frac{\beta}{\sqrt{6}}|1\rangle(2|10\rangle +|11\rangle + |01\rangle) \\
|\Psi_2\rangle &= \frac{\alpha}{\sqrt{6}}|0\rangle(2|00\rangle +|01\rangle + |11\rangle) + \frac{\beta}{\sqrt{6}}|1\rangle(2|11\rangle +|10\rangle + |00\rangle) \\
&= \sqrt{\frac{2}{3}}(\alpha|000\rangle+\beta|111\rangle) + \frac{1}{\sqrt{6}}(\alpha|001\rangle +\alpha|011\rangle + \beta|110\rangle + \beta |100\rangle) \\
|\Psi_3\rangle &= \sqrt{\frac{2}{3}}(\alpha|000\rangle+\beta|011\rangle) + \frac{1}{\sqrt{6}}(\alpha|001\rangle +\alpha|111\rangle + \beta|010\rangle + \beta |100\rangle) \\
|\Psi_4\rangle &= \sqrt{\frac{2}{3}}(\alpha|000\rangle+\beta|111\rangle) + \frac{1}{\sqrt{6}}(\alpha|101\rangle +\alpha|011\rangle + \beta|010\rangle + \beta |100\rangle) \\
&=\left( \sqrt{\frac{2}{3}}\alpha|00\rangle +\beta \frac{1}{\sqrt{6}}(|10\rangle+|01\rangle)\right)|0\rangle\notag \\
&\qquad \qquad \qquad \qquad \qquad \qquad + \left( \sqrt{\frac{2}{3}}\beta|11\rangle + \frac{1}{\sqrt{6}}\alpha(|10\rangle + |01\rangle )\right)|1\rangle \\
&= |\chi_0\rangle|0\rangle+|\chi_1\rangle|1\rangle\equiv \ket{\Psi},
\end{align}
where we have defined
\begin{align}
|\chi_0\rangle &= \sqrt{\frac{2}{3}}\alpha|00\rangle + \sqrt{\frac{1}{3}}\beta|\Phi_+\rangle=\sqrt{\frac{2}{3}}\alpha\ket{00}+\sqrt{\frac{1}{6}}\beta\ket{01}+\sqrt{\frac{1}{6}}\beta\ket{10}\\
|\chi_1\rangle &= \sqrt{\frac{2}{3}}\beta|11\rangle +\sqrt{\frac{1}{3}}\alpha |\Phi_+\rangle=\sqrt{\frac{2}{3}}\beta\ket{11}+\sqrt{\frac{1}{6}}\alpha\ket{01}+\sqrt{\frac{1}{6}}\alpha\ket{10}
\end{align}

\subsection{Single state fidelity}
\[
    \ket{\Psi}=\ket{\chi_0}\ket{0}+\ket{\chi_1}\ket{1}=\left(\sqrt{\frac{2}{3}}\alpha\ket{00}+\sqrt{\frac{1}{3}}\beta\ket{\Phi_+}\right)\ket{0}+\left(\sqrt{\frac{2}{3}}\beta\ket{11}+\sqrt{\frac{1}{3}}\alpha\ket{\Phi_+}\right)\ket{1}\\
\]
\[
    \rho_{a_0,a_1}=\Tr_{b_1}[\ketbra{\psi_4}{\psi_4}]=\ketbra{\chi_0}{\chi_0}+\ketbra{\chi_1}{\chi_1}=
\]
Note that both $\ket{\chi_0}$ and $\ket{\chi_1}$ are invariant under exchange of the qubits. This implies that $\rho_{a_0}=\rho_{a_1}$, i.e. the copies are identical (there is no different in calculating the partial trace in one Hilbert space or the other). Hence:
\[
    \begin{split}
    \rho_{a_0}&=\Tr_{a_1}[\ketbra{\chi_0}{\chi_0}+\ketbra{\chi_1}{\chi_1}]=\Tr_{a_1}[\ketbra{\chi_0}{\chi_0}]+\Tr_{a_1}[\ketbra{\chi_1}{\chi_1}]=\\
    &=\frac{2}{3}\abs{\alpha}^2\ketbra{0}{0}+\frac{1}{6}\abs{\beta}^2\ketbra{0}{0}+\frac{1}{6}\abs{\beta}^2\ketbra{1}{1}+\frac{1}{3}\alpha\beta^*\ketbra{0}{1}+\frac{1}{3}\alpha^*\beta\ketbra{1}{0}+\\
    &+\frac{2}{3}\abs{\beta}^2\ketbra{1}{1}+\frac{1}{6}\abs{\alpha}^2\ketbra{0}{0}+\frac{1}{6}\abs{\alpha}^2\ketbra{1}{1}+\frac{1}{3}\alpha\beta^*\ketbra{0}{1}+\frac{1}{3}\alpha^*\beta\ketbra{1}{0}=\\
    &=\frac{5}{6}\abs{\alpha}^2\ketbra{0}{0}+\frac{5}{6}\abs{\beta}^2\ketbra{1}{1}+\frac{1}{6}\abs{\beta}^2\ketbra{0}{0}+\frac{1}{6}\abs{\alpha}^2\ketbra{1}{1}+\frac{2}{3}\alpha\beta^*\ketbra{0}{1}+\frac{2}{3}\alpha^*\beta\ketbra{1}{0}=\\
    &=\frac{5}{6}\abs{\alpha}^2\ketbra{0}{0}+\frac{5}{6}\abs{\beta}^2\ketbra{1}{1}+\frac{5}{6}\alpha\beta^*\ketbra{0}{1}+\frac{5}{6}\alpha^*\beta\ketbra{1}{0}-\\
    &-\frac{1}{6}\alpha\beta^*\ketbra{0}{1}-\frac{1}{6}\alpha^*\beta\ketbra{1}{0}+\frac{1}{6}\abs{\beta}^2\ketbra{0}{0}+\frac{1}{6}\abs{\alpha}^2\ketbra{1}{1}=\\
    &=\frac{5}{6}\ketbra{\psi}{\psi}+\frac{1}{6}\ketbra{\psi_{\perp}}{\psi_{\perp}},
    \end{split}
\]
where $\ket{\psi}=\alpha\ket{0}+\beta\ket{1}$ is the input state and $\ket{\psi_{\perp}}=\beta^*\ket{0}-\alpha^*\ket{1}$ is its orthogonal state.

A common figure of merit used in order to estimate the closeness of two quantum states is the fidelity [Ref: Nielsen and Chuang]

\begin{equation}
    F(\ket{\psi},\rho)=\bra{\psi}\rho\ket{\psi}
\end{equation}

In our case, the fidelity of the output copies is:
\[
    F(\ket{\psi},\rho_{a_0})=\bra{\psi}\rho_{a_0}\ket{\psi}=\frac{5}{6}= F(\ket{\psi},\rho_{a_1})
\]

\section{Implementation of UQCM in QI}
In order to run the circuit on Starmon-5, we have to consider that we can only apply $2$-qubits gate on nearest neighbours. 
In order to overcome this problem, we have introduced two SWAP gates (this is not the best choice probably, it can be done with only one).

\begin{quantikz}
    \lstick{$|0\rangle_{a_1}$} & \gate{R_y(\theta_1)}  & \swap{2}& \qw & \qw & \qw &\swap{2} &\gate{R_y(\theta_3)} & \targ{}&\qw&\ctrl{2}&\qw & \meter{}\\
    \lstick{} & \qw & \qw& \qw& \qw& \qw& \qw& \qw & \qw& \qw& \qw & \qw& \qw \\
    \lstick{$|\psi \rangle_{a_0}$} & \qw & \targX{} & \ctrl{2} & \qw & \targ{} & \targX{} & \qw & \ctrl{-2} & \ctrl{2} & \targ{} & \targ{} & \qw \\
    \lstick{} & \qw & \qw& \qw& \qw& \qw& \qw& \qw & \qw& \qw& \qw & \qw& \qw \\
    \lstick{$|0\rangle_{b_1}$} & \qw & \qw & \targ{} & \gate{R_y(\theta_2)}  &\ctrl{-2} & \qw & \qw & \qw & \targ{} & \qw & \ctrl{-2} & \qw \\
\end{quantikz}.
The generic input state $\ket{\psi}$ can be prepared from $\ket{0}$ performing two rotations: 
\[
    \ket{\psi}_{a_0}=R_z(\phi)R_y(\theta)\ket{0}
\]

The final measurement is performed in the $\{\ket{\psi},\ket{\psi_{\perp}}\}$ basis. The probability of getting $+1$ as outcome is:
\[
    p_{+1}=\Tr[\ketbra{\psi}{\psi}\rho_{a_1}]=\bra{\psi}\rho_{a_1}\ket{\psi}=F(\ket{\psi},\rho_{a_1}).
\]
Hence, we can measure the fidelity directly, without having to perform a quantum tomography experiment.
In order to perform this measurement, we implement the inverse rotations with respect to the ones used to prepare the input state. 
Hence, we first rotate with $R_z(-\phi)$ and then with $R_y(-\theta)$. We conclude by measuring in the computational basis.

\subsection*{QXSimulator}
In order to verify that our implementation of the AQCM is correct, we first ran the experiment on the QXSimulator, considering $1000$ points on the Bloch sphere and $1024$ shots for each point.
NB: In the following pictures $F_{measured}/F_{optimal}$ is plotted.
\begin{figure}[H]
    \centering
            \includegraphics[totalheight=6cm]{Figures/simulator_F.png}
        \label{fig:2a}
\end{figure}
When averaging over the Bloch sphere, we obtain
\[
    \overline{F}=0.833 \pm 0.011.
\]
This is consistent with the expected value $F_{optimal}=\frac{5}{6}=0.833\dots$
\subsection*{Starmon-5}
Afterwards, we ran the circuit on Starmon-5. We sampled the Bloch sphere using $1000$ points and considering $4096$ shot for each point. 
NB: In the following pictures $F_{measured}/F_{optimal}$ is plotted.
\begin{figure}[H]
    \centering
            \includegraphics[totalheight=6cm]{Figures/starmon(1,1)_F.png}
        \label{fig:2a}
\end{figure}
\begin{figure}[H]
    \centering
            \includegraphics[totalheight=6cm]{Figures/starmon(-1,-1)_F.png}
        \label{fig:2a}
\end{figure}
\begin{figure}[H]
    \centering
            \includegraphics[totalheight=6cm]{Figures/starmon(1,-1)_F.png}
        \label{fig:2a}
\end{figure}
\begin{figure}[H]
    \centering
            \includegraphics[totalheight=6cm]{Figures/starmon(-1,1)_F.png}
        \label{fig:2a}
\end{figure}
When averaging over the Bloch sphere, we obtain
\[
    \overline{F}=0.75 \pm 0.03.
\]
As expected, this is lower than the optimal fidelity. When running on real hardware, noise will reduce the quality of the copies.
\newpage
\end{comment}
\section{Report 2}
\subsection{Updated circuit}
We updated the circuit considering only 1 SWAP gate:

\begin{quantikz}
    \lstick{$\ket{\psi}_{a_0}$} & \qw                  & \qw        & \qw                   & \qw       & \qw                   & \swap{2}  & \targ{}   & \qw       & \ctrl{2} & \qw        & \meter{}\\
    \lstick{$\ket{0}_{b_1}$}    & \qw                  & \targ{}    & \gate{R_y(\theta_2)}  & \ctrl{1}  & \qw                   & \qw       & \qw       & \targ{}   & \qw      & \ctrl{1}   & \qw         \\
    \lstick{$\ket{0}_{a_1}$}    & \gate{R_y(\theta_1)} & \ctrl{-1}  & \qw                   & \targ{}   & \gate{R_y(\theta_3)}  & \targX{}  & \ctrl{-2} & \ctrl{-1} & \targ{}  & \targ{}    & \meter{}   \\
    \lstick{}\\
    \lstick{}\\
\end{quantikz}


This is also how IBM transpiles the original circuit proposed in [Ref Buzek-Hillery] in order to run it with only nearest neighbour couplings.

We are now measuring both the copies. We are checking whether the QCM is actually symmetric.
\subsection{Results on Starmon}
See Jupyter.
\subsection{Results on IBM}
See Jupyter.
We turned off the optimization in the transpiling process. The transpiler only adds the SWAP gate in order to allow the circuit to run when only nearest neighbour coupling are allowed (usually it is not possible to have three qubits reciprocally connected).
When using the Yorktown backend, no SWAP gate is necessary.
\begin{figure}[H]
    \centering
            \includegraphics[totalheight=6cm]{Figures/yorktownconnectivity.png}
        \label{fig:2a}\caption{Connectivity of Yorktown IBM device.}
\end{figure}

\subsection{Comparison}
\begin{table}[H]
    \centering
    \begin{tabular}{|c|c|c|c|c|c|c|c|}
    \hline
    \textbf{} & \textbf{Starmon 5} & \textbf{Athens} & \textbf{Ourense} & \textbf{Santiago} & \textbf{Valencia} & \textbf{Vigo} & \textbf{Yorktown} \\ \hline
    $F_1$              & 0.73 & 0.77 & 0.78 & 0.73 & 0.6 & 0.76 & 0.76 \\ \hline
    $\sigma_{F_1}$     & 0.03 & 0.02 & 0.03 & 0.04 & 0.2 & 0.03 & 0.02 \\ \hline
    $F_2$              & 0.73 & 0.78 & 0.72 & 0.72 & 0.6 & 0.75 & 0.76 \\ \hline
    $\sigma_{F_2}$     & 0.05 & 0.02 & 0.03 & 0.04 & 0.2 & 0.03 & 0.02 \\ \hline
    $F_2/F_1$          & 1.00 & 1.01 & 0.93 & 0.98 & 1.1 & 0.99 & 1.00 \\ \hline
    $\sigma_{F_2/F_1}$ & 0.04 & 0.03 & 0.03 & 0.03 & 0.7 & 0.02 & 0.03 \\ \hline
    \end{tabular}
    \end{table}

\subsection{Readout calibration}
We considered that a classical error can affect the measurement process. Consider a state of the form $\ket{\psi} = \alpha\ket{0}+\beta\ket{1}$. The expectation value of a measurement along the Z axis is given by $\bar{m}=p_{+1}-p_{-1}$. However, this quantity can be affected by classical errors, such that it is transformed to,
\begin{align}
\bar{m} &= (1-2\epsilon_{10})|\alpha|^2-(1-2\epsilon_{01})|\beta|^2\\
&=(\epsilon_{01}-\epsilon_{10})+(1-\epsilon_{01}-\epsilon_{10})(|\alpha|^2-|\beta|^2)\\
&= \beta_0 + \beta_1 \langle Z\rangle.
\end{align}
Note that $|\alpha|^2=p_{+1}$ and $|\beta|^2=p_{-1}$. Then, we can find the corrected expectation value as,
\begin{equation}
\langle Z\rangle = \frac{\bar{m}-\beta_0}{\beta_1}.
\end{equation}
The parameters $\beta_0$ and $\beta_1$ can be estimated from experimental measurements of eigenstates of the Z basis, that is, $\ket{0}$ and $\ket{1}$. Then, they can be used to correct the outcome of future experiments, for example a quantum state tomography. In Fig. \ref{fig:3a} one can observe a check that the results have been calibrated. It corresponds to the average $Z$ value of a single qubit, when it is rotated along the $X$ axis from $0$ to $2\pi$. The blue curve represents the measurement outcomes without any correction, while the orange curve represents the corrected measurements. 
\begin{figure}[H]
    \centering
            \includegraphics[totalheight=6cm]{Figures/tomography.png}
        \label{fig:3a}\caption{Calibration check on qubit 0 of Starmon five. The calibration parameters were estimated to be $\beta_0 = 0.1229$ and $\beta_1=0.8278$. Blue curve corresponds to the pure results. Orange curve corresponds to the corrected results.}
\end{figure}
We are not only interested in corrected the expectation value, but also the coefficients associated to each measurement outcome. For such purpose, we consider corrected coefficients $\tilde{p_{+1}}$ and $\tilde{p_{-1}}$ that satisfy the following equations,
\begin{equation}
\left\{ \begin{array}{c}
\tilde{p_{+1}} + \tilde{p_{-1}} = 1\\
\tilde{p_{+1}} - \tilde{p_{-1}} = \frac{\bar{m}-\beta_0}{\beta_1}
\end{array} \right.
\end{equation}
By solving this system of equations, we find that the corrected coefficients are given by
\begin{equation}
\tilde{p_{+1}} = \frac{\beta_1-\beta_0+p_{+1}-p_{-1}}{2\beta_1},\qquad \tilde{p_{-1}} = \frac{\beta_1+\beta_0-p_{+1}+p_{-1}}{2\beta_1}.
\end{equation} 
From this parameters we can correct the previous results corresponding to the average fidelity of the quantum copy machine on the Bloch sphere. 
\subsection{Next meetings}
\begin{enumerate}
    \item Readout correction on IBM
    \item Readout correction for all the five qubits of starmon.
    \item Phase covariant cloning machine (optimal copy of the states on one equator)
    \item Economical cloning machine (2 qubits, we could run it on Spin-2!)
    \item Asymmetric Fourier cloning machine (?)
\end{enumerate}
\end{document}
