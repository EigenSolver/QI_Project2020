\chapter{Results}
\label{sec:results}

In this section we list and discuss the results obtained.
It is worth mentioning that calibrations and/or modifications of the quantum processors might change significantly the results obtained here.

\section{Universal quantum cloning machine}
\subsection{Sphere}

In Sec. \ref{sec:approximateqcm} we found that the UQCM returns a fidelity of $F=5/6$ for all the states in the Bloch sphere. The results that we found are summarized in Table \ref{tab:uqcm_results_full_sphere}. 

\begin{table}[H]
    \centering
    \begin{tabular}{|c|c|c|c|c|c|c|c|}
    \hline
    \textbf{} & \textbf{Starmon-5} & \textbf{Athens} & \textbf{Ourense} & \textbf{Santiago} & \textbf{Valencia} & \textbf{Vigo} & \textbf{Yorktown} \\ \hline
    $F_1$              &  &  & 0.777 & 0.777 & 0.752 & 0.687 & 0.752\\ \hline
    $\sigma_{F_1}$     &  &  & 0.029 & 0.025 & 0.091 & 0.036 & 0.025 \\ \hline
    $F_2$              &  &  & 0.771 & 0.732 & 0.724 & 0.714 & 0.762 \\ \hline
    $\sigma_{F_2}$     &  &  & 0.035 & 0.030 & 0.064 & 0.068 & 0.017 \\ \hline
    \end{tabular}
    \caption{Average fidelity and corresponding standard deviation on the whole Bloch sphere for the two copies on each backend. The results are corrected with readout calibration.}\label{tab:uqcm_results_full_sphere}
\end{table}

In general, we find that the first copy performs better than the second one for all the cases. Also, the fidelity is lower than the expected one, and even lower than other copy machines (as will be discussed later), which are supposed to be worse. The standart deviations are close to the theoretical expectation. We found that the backend that have the highest fidelities is Ourense from IBM. The backend with the lowest standart deviation is Yorktown from IBM. We attirbute the later to the connectivity found in this device, which maps idealy the one of the UQCM. Nevertheless, this circuit is has a high cost of CNOT gates, which significantly increases the error in the fidelity.

\begin{figure}[H]
    \centering
    \begin{subfigure}{.5\textwidth}
      \centering
      \includegraphics[width=\textwidth]{Figures/UQCM/IBM/FullSphere/results_corrected_athens_copy1.png}
    \end{subfigure}%
    \begin{subfigure}{.5\textwidth}
      \centering
      \includegraphics[width=\textwidth]{Figures/UQCM/IBM/FullSphere/results_corrected_athens_copy2.png}
    \end{subfigure}
    \caption{Fidelity sampled over the Bloch sphere using the UQCM implemented on Athens using readout correction.}
    \label{fig:uqcm_athens}
\end{figure}


\begin{figure}[H]
    \centering
    \begin{subfigure}{.5\textwidth}
      \centering
      \includegraphics[width=\textwidth]{Figures/UQCM/IBM/FullSphere/results_corrected_ourense_copy1.png}
    \end{subfigure}%
    \begin{subfigure}{.5\textwidth}
      \centering
      \includegraphics[width=\textwidth]{Figures/UQCM/IBM/FullSphere/results_corrected_ourense_copy2.png}
    \end{subfigure}
    \caption{Fidelity sampled over the Bloch sphere using the UQCM implemented on Ourense using readout correction.}
    \label{fig:uqcm_our}
\end{figure}


\begin{figure}[H]
    \centering
    \begin{subfigure}{.5\textwidth}
      \centering
      \includegraphics[width=\textwidth]{Figures/UQCM/IBM/FullSphere/results_corrected_vigo_copy1.png}
    \end{subfigure}%
    \begin{subfigure}{.5\textwidth}
      \centering
      \includegraphics[width=\textwidth]{Figures/UQCM/IBM/FullSphere/results_corrected_vigo_copy2.png}
    \end{subfigure}
    \caption{Fidelity sampled over the Bloch sphere using the UQCM implemented on Vigo using readout correction.}
    \label{fig:uqcm_vigo}
\end{figure}

\begin{figure}[H]
    \centering
    \begin{subfigure}{.5\textwidth}
      \centering
      \includegraphics[width=\textwidth]{Figures/UQCM/IBM/FullSphere/results_corrected_valencia_copy1.png}
    \end{subfigure}%
    \begin{subfigure}{.5\textwidth}
      \centering
      \includegraphics[width=\textwidth]{Figures/UQCM/IBM/FullSphere/results_corrected_valencia_copy2.png}
    \end{subfigure}
    \caption{Fidelity sampled over the Bloch sphere using the UQCM implemented on Valencia using readout correction.}
    \label{fig:uqcm_vigo}
\end{figure}



\begin{figure}[H]
    \centering
    \begin{subfigure}{.5\textwidth}
      \centering
      \includegraphics[width=\textwidth]{Figures/UQCM/IBM/FullSphere/results_corrected_santiago_copy1.png}
    \end{subfigure}%
    \begin{subfigure}{.5\textwidth}
      \centering
      \includegraphics[width=\textwidth]{Figures/UQCM/IBM/FullSphere/results_corrected_santiago_copy2.png}
    \end{subfigure}
    \caption{Fidelity sampled over the Bloch sphere using the UQCM implemented on Santiago using readout correction.}
    \label{fig:uqcm_sant}
\end{figure}

\begin{figure}[H]
    \centering
    \begin{subfigure}{.5\textwidth}
      \centering
      \includegraphics[width=\textwidth]{Figures/UQCM/IBM/FullSphere/results_corrected_ibmqx2_copy1.png}
    \end{subfigure}%
    \begin{subfigure}{.5\textwidth}
      \centering
      \includegraphics[width=\textwidth]{Figures/UQCM/IBM/FullSphere/results_corrected_ibmqx2_copy2.png}
    \end{subfigure}
\caption{Fidelity sampled over the Bloch sphere using the UQCM implemented on Yorktown using readout correction.}
    \label{fig:uqcm_york}
\end{figure}

We are not expecting any recognizable pattern in the Bloch sphere since the fidelity should be equal for all single qubit states. This is the case for most of the backends. We found an almost uniform sampling along the equatorial region as shown in Figs. \ref{fig:uqcm_athens}, \ref{fig:uqcm_our}, \ref{fig:uqcm_sant}, \ref{fig:uqcm_vigo}. In Fig. \ref{fig:uqcm_york} we find a recognizable pattern which improves the fidelity along the axis of the Bloch sphere. We may attribute this to the internal working of the Yorktown machine rather than to any specific feature of the circuit.

For completeness, we include the results without the readout correction in Table \ref{tab:uqcm_results_full_sphere_non}. We can appreciate that the values have sightly changed. In general, we find that the readout correction improves the fidelity of one of the two copies, while decreases the fidelity of the other copy. We do not expect always an improvement, but a results closer to the real output of the machine, which already includes errors from other sources during the execution.
 
\begin{table}[H]
    \centering
    \begin{tabular}{|c|c|c|c|c|c|c|c|}
    \hline
    \textbf{} & \textbf{Starmon-5} & \textbf{Athens} & \textbf{Ourense} & \textbf{Santiago} & \textbf{Valencia} & \textbf{Vigo} & \textbf{Yorktown} \\ \hline
    $F_1$              &  &  & 0.777 & 0.777 & 0.752 & 0.687 & 0.752\\ \hline
    $\sigma_{F_1}$     &  &  & 0.029 & 0.025 & 0.091 & 0.036 & 0.025 \\ \hline
    $F_2$              &  &  & 0.771 & 0.732 & 0.724 & 0.714 & 0.762 \\ \hline
    $\sigma_{F_2}$     &  &  & 0.035 & 0.030 & 0.064 & 0.068 & 0.017 \\ \hline
    \end{tabular}
    \caption{Average fidelity and corresponding standard deviation on the whole Bloch sphere for the two copies on each backend. The
results are not corrected with readout calibration.}\label{tab:uqcm_results_full_sphere_non}
\end{table}

\subsection{Equator}

For a sample of 100 points along the xz equator of the Bloch sphere we run the UQCM and obtain the results listed in Table \ref{tab:uqcm_results_eq}. We found that the fidelity has improved with respect to the full sphere case for both copies. Since the fidelity presented here is an average over all the sampled points, it is not expected to have an improvement with respect to the previous case. Nevertheless, preparing the qubit in a state along the equator involves less operations than preparing it in an arbitrary state. Then, it is possible to expect an improvement in the fidelities.

\begin{table}[H]
    \centering
    \begin{tabular}{|c|c|c|c|c|c|c|c|}
    \hline
    \textbf{} & \textbf{Starmon-5} & \textbf{Athens} & \textbf{Ourense} & \textbf{Santiago} & \textbf{Valencia} & \textbf{Vigo} & \textbf{Yorktown} \\ \hline
    $F_1$              & 0.747 & 0.792 & 0.790 &  &  & 0.729 & 0.737\\ \hline
    $\sigma_{F_1}$     & 0.026 & 0.016 & 0.031 &  &  & 0.033 & 0.020 \\ \hline
    $F_2$              & 0.719 & 0.793 & 0.795 &  &  & 0.736 & 0.763 \\ \hline
    $\sigma_{F_2}$     & 0.039 & 0.020 & 0.036 &  &  & 0.074 & 0.020 \\ \hline
    \end{tabular}
    \caption{Average fidelity and corresponding standard deviation on the $xz$ equator of the Bloch sphere for the two copies on each backend. The results are corrected with readout calibration.}\label{tab:uqcm_results_eq}
\end{table}

In the following figures, we observe the fidelity of the states along the equator. We found that the fidelity of these states has a oscillatory behaviour. We expect it to be uniform, as is the case for Athens in Fig. \ref{fig:uqcm_eq_athens}. However, most of the previous figures show a clear oscillatory pattern. We attribute this behaviour to the error induced during the preparation of the states.

\begin{figure}[H]
    \centering
    \begin{subfigure}{.45\textwidth}
      \centering
      \includegraphics[width=\textwidth]{Figures/UQCM/Starmon/OnlyEquator/results_corrected_starmon5_copy1.png}
    \end{subfigure}%
    \begin{subfigure}{.45\textwidth}
      \centering
      \includegraphics[width=\textwidth]{Figures/UQCM/Starmon/OnlyEquator/results_corrected_starmon5_copy2.png}
    \end{subfigure}
    \caption{Fidelity sampled over the $xz$-equator using the UQCM implemented on Starmon-5 using readout correction. The solid green line is the expected fidelity, the solid blue line is the measured average fidelity, the dashed red lines indicate the expected dispersion and the dashed blue line indicate the measured dispersion. The orange points are the BB84 states.}
    \label{fig:uqcm_eq_starmon}
\end{figure}


\begin{figure}[H]
    \centering
    \begin{subfigure}{.45\textwidth}
      \centering
      \includegraphics[width=\textwidth]{Figures/UQCM/IBM/OnlyEquator/results_corrected_ourense_copy1.png}
    \end{subfigure}%
    \begin{subfigure}{.45\textwidth}
      \centering
      \includegraphics[width=\textwidth]{Figures/UQCM/IBM/OnlyEquator/results_corrected_ourense_copy2.png}
    \end{subfigure}
    \caption{Fidelity sampled over the $xz$-equator using the UQCM implemented on Ourense using readout correction. The solid green line is the expected fidelity, the solid blue line is the measured average fidelity, the dashed red lines indicate the expected dispersion and the dashed blue line indicate the measured dispersion. The orange points are the BB84 states.}
    \label{fig:uqcm_eq_our}
\end{figure}

\begin{figure}[H]
    \centering
    \begin{subfigure}{.45\textwidth}
      \centering
      \includegraphics[width=\textwidth]{Figures/UQCM/IBM/OnlyEquator/results_corrected_athens_copy1.png}
    \end{subfigure}%
    \begin{subfigure}{.45\textwidth}
      \centering
      \includegraphics[width=\textwidth]{Figures/UQCM/IBM/OnlyEquator/results_corrected_athens_copy2.png}
    \end{subfigure}
    \caption{Fidelity sampled over the $xz$-equator using the UQCM implemented on Athens using readout correction. The solid green line is the expected fidelity, the solid blue line is the measured average fidelity, the dashed red lines indicate the expected dispersion and the dashed blue line indicate the measured dispersion. The orange points are the BB84 states.}
    \label{fig:uqcm_eq_athens}
\end{figure}


\begin{figure}[H]
    \centering
    \begin{subfigure}{.45\textwidth}
      \centering
      \includegraphics[width=\textwidth]{Figures/UQCM/IBM/OnlyEquator/results_corrected_vigo_copy1.png}
    \end{subfigure}%
    \begin{subfigure}{.45\textwidth}
      \centering
      \includegraphics[width=\textwidth]{Figures/UQCM/IBM/OnlyEquator/results_corrected_vigo_copy2.png}
    \end{subfigure}
    \caption{Fidelity sampled over the $xz$-equator using the UQCM implemented on Vigo using readout correction. The solid green line is the expected fidelity, the solid blue line is the measured average fidelity, the dashed red lines indicate the expected dispersion and the dashed blue line indicate the measured dispersion. The orange points are the BB84 states.}
    \label{fig:uqcm_eq_vigo}
\end{figure}


\begin{figure}[H]
    \centering
    \begin{subfigure}{.45\textwidth}
      \centering
      \includegraphics[width=\textwidth]{Figures/UQCM/IBM/OnlyEquator/results_corrected_ibmqx2_copy1.png}
    \end{subfigure}%
    \begin{subfigure}{.45\textwidth}
      \centering
      \includegraphics[width=\textwidth]{Figures/UQCM/IBM/OnlyEquator/results_corrected_ibmqx2_copy2.png}
    \end{subfigure}
    \caption{Fidelity sampled over the $xz$-equator using the UQCM implemented on Yorktown using readout correction. The solid green line is the expected fidelity, the solid blue line is the measured average fidelity, the dashed red lines indicate the expected dispersion and the dashed blue line indicate the measured dispersion. The orange points are the BB84 states.}
    \label{fig:uqcm_eq_york}
\end{figure}

For completeness, we include the results without the readout correction. The results are shown in Table \ref{tab:uqcm_results_eq_non}. As before, we find that the readout improves the readout in few cases, e.g. Athens, and it decreases in most of them. The results are within the statistical uncertanty attributed to the finite sampling.

\begin{table}[H]
    \centering
    \begin{tabular}{|c|c|c|c|c|c|c|c|}
    \hline
    \textbf{} & \textbf{Starmon-5} & \textbf{Athens} & \textbf{Ourense} & \textbf{Santiago} & \textbf{Valencia} & \textbf{Vigo} & \textbf{Yorktown} \\ \hline
    $F_1$              & 0.741 & 0.793 & 0.788 &  &  & 0.741 & 0.748\\ \hline
    $\sigma_{F_1}$     & 0.025 & 0.016 & 0.029 &  &  & 0.030 & 0.018 \\ \hline
    $F_2$              & 0.721 & 0.794 & 0.790 &  &  & 0.743 & 0.762 \\ \hline
    $\sigma_{F_2}$     & 0.038 & 0.019 & 0.036 &  &  & 0.070 & 0.019 \\ \hline
    \end{tabular}
    \caption{Average fidelity and corresponding standard deviation on the $xz$ equator of the Bloch sphere for the two copies on each backend. The results are not corrected with readout calibration.}\label{tab:uqcm_results_eq_non}
\end{table}

\subsection{BB84 States}

We have sampled the BB84 states corresponding to the states pointing along the $X$ and $Z$ axis of the Bloch sphere. The results are shown in Table \ref{tab:uqcm_results_bb} with readout correction, and in Table \ref{sec:results} without it. We omitted the graphics since the can be found as orange points in the graphics of the UCQM sampled along the equator.

\begin{table}[H]
    \centering
    \begin{tabular}{|c|c|c|c|c|c|c|c|}
    \hline
    \textbf{} & \textbf{Starmon-5} & \textbf{Athens} & \textbf{Ourense} & \textbf{Santiago} & \textbf{Valencia} & \textbf{Vigo} & \textbf{Yorktown} \\ \hline
    $F_1$              & 0.755 &  & 0.837 &  & 0.690 & 0.730 & 0.760\\ \hline
    $\sigma_{F_1}$     & 0.035 &  & 0.040 &  & 0.034 & 0.0475 & 0.012 \\ \hline
    $F_2$              & 0.720 &  & 0.765 &  & 0.690 & 0.735 & 0.761 \\ \hline
    $\sigma_{F_2}$     & 0.041 &  & 0.065 &  & 0.045 & 0.052 & 0.026 \\ \hline
    \end{tabular}
    \caption{Average fidelity and corresponding standard deviation on the BB84 states for the two copies on each backend. The results are corrected with readout calibration.}\label{tab:uqcm_results_bb}
\end{table}

We observe that in most of the cases the fidelity is decreased by the readout correction. We found that Ourense performs the best for a copy of the BB84 states, while Vigo is the worst. These features are atributted to the details of each specific device. Nevertheless they would be suitable to simulate a noisy channel where the copied qubit has decreased its fidelity to the values presented here.

\begin{table}[H]
    \centering
    \begin{tabular}{|c|c|c|c|c|c|c|c|}
    \hline
    \textbf{} & \textbf{Starmon-5} & \textbf{Athens} & \textbf{Ourense} & \textbf{Santiago} & \textbf{Valencia} & \textbf{Vigo} & \textbf{Yorktown} \\ \hline
    $F_1$              & 0.755 &  & 0.835 &  & 0.690 & 0.762 & 0.772\\ \hline
    $\sigma_{F_1}$     & 0.034 &  & 0.038 &  & 0.027 & 0.040 & 0.011 \\ \hline
    $F_2$              & 0.719 &  & 0.777 &  & 0.699 & 0.742 & 0.764 \\ \hline
    $\sigma_{F_2}$     & 0.039 &  & 0.057 &  & 0.042 & 0.049 & 0.026 \\ \hline
    \end{tabular}
    \caption{Average fidelity and corresponding standard deviation on the BB84 states for the two copies on each backend. The results are not corrected with readout calibration.}\label{tab:uqcm_results_bb_non}
\end{table}
\section{Phase covariant quantum cloning machine}
\subsection{Sphere}
In Chapter \ref{sec:approximateqcm}, we found that the expected average fidelity and the associated standard deviation on the whole Bloch sphere are (see Equation \ref{eqn:avg_fidelity_pcqcm} and Equation \ref{eqn:std_fidelity_pcqcm}):
\[
  \overline{F}\approx0.819, \quad \sigma_{F}\approx0.031.
\]

The results we found are summarized in Table \ref{tab:results_pcqcm_fullsphere_corrected} (corrected with readout calibration).
\begin{table}[H]
    \centering
    \begin{tabular}{|c|c|c|c|c|c|c|c|}
    \hline
    \textbf{} & \textbf{Starmon-5} & \textbf{Athens} & \textbf{Ourense} & \textbf{Santiago} & \textbf{Valencia} & \textbf{Vigo} & \textbf{Yorktown} \\ \hline
    $F_1$              & 0.774 & 0.779 & 0.802 & 0.796 & 0.800 & 0.801 & 0.794 \\ \hline
    $\sigma_{F_1}$     & 0.041 & 0.036 & 0.031 & 0.035 & 0.040 & 0.035 & 0.025 \\ \hline
    $F_2$              & 0.766 & 0.823 & 0.780 & 0.773 & 0.791 & 0.807 & 0.755 \\ \hline
    $\sigma_{F_2}$     & 0.041 & 0.031 & 0.034 & 0.045 & 0.034 & 0.036 & 0.042 \\ \hline
    \end{tabular}
    \caption{Average fidelity and corresponding standard deviation on the whole Bloch sphere for the two copies on each backend. The results are corrected with readout calibration.}
    \label{tab:results_pcqcm_fullsphere_corrected}
\end{table}
\begin{figure}[H]
  \centering
          \includegraphics[width=0.8\textwidth]{Figures/PhaseCovariant/Histograms/histo_sphere_corrected.png}
      \label{fig:pc_histo_sphere_corrected}
      \caption{Comparison of the average fidelity on the Bloch sphere for both the copies on different backends using the PCQCM with readout correction. The uncertainty on the average is calculated as $\sigma_{\overline{F}_i}=\sigma_{F_i}/\sqrt{N}$, where $N$ is the number of points sampled ($N=1000$ in this case).}
\end{figure}

It is possible to observe that, on average, the PCQCM performs better than the UQCM, even if theoretically it should perform worse. 
As mentioned in Chapter \ref{sec:approximateqcm}, this is because the PCQCM circuit is much simpler than the UQCM one.
The dispersion of the results is close to the expected one, however the average fidelity is lower than the ideal one (with the exception of the fidelity of the second copy for Athens).
This suggests that the expected fidelity of the ideal PCQCM on the Bloch sphere is lowered more or less equally across the Bloch sphere, so that the qualitative pattern is preserved.
Indeed, in the pictures below, it is possible to observe that the $xz$-equator performs better than the rest of the two emisphere (the darker lobes).

In the following we report the projection of the results obtained on the Bloch sphere, corrected with the readout calibration (the non-corrected results can be observed in the appendix). 
In general, we can see the expected pattern for the fidelity.
In certain cases, for example for Starmon-5 (Figure \ref{fig:pc_corrected_starmon_sphere}), the results are not symmetric: the region around $\ket{-}$ (i.e. the left lobe) is copied better than the region around $\ket{+}$ (i.e. the right lobe) on both the qubits (this happens in both the copies).
Sometimes, for example looking at the results for Santiago (Figure \ref{fig:pc_corrected_santiago_sphere}), it is possible to observe some sudden discontinuities.
These artefacts, present both in the corrected and the non-corrected results, correspond to one batch of circuits (on the IBM backends it is possible to group $75$ circuits together).
This could be due to the fact that two subsequent batches of circuits could be run hours apart, depending on the length of the queue.
Therefore, it should not be assumed that the condition and the calibration of the quantum processors are the same throughout the sampling.
The quality of the results could be improved if we could run all the circuits together.
Finally, we observe again that the results obtained on Yorktown (\ref{fig:pc_corrected_yorktown_sphere}) are quite different compared to the others, even if in this case the connectivity of the quantum processor should not play a role.

\begin{figure}[H]
  \centering
  \begin{subfigure}{.5\textwidth}
    \centering
    \includegraphics[width=\textwidth]{Figures/PhaseCovariant/Starmon/FullSphere/results_starmon_corrected_copy1.png}
    \label{fig:pc_corrected_starmon_sphere_1}
  \end{subfigure}%
  \begin{subfigure}{.5\textwidth}
    \centering
    \includegraphics[width=\textwidth]{Figures/PhaseCovariant/Starmon/FullSphere/results_starmon_corrected_copy2.png}
    \label{fig:pc_corrected_starmon_sphere_2}
  \end{subfigure}
  \caption{Fidelity sampled over the Bloch sphere using the PCQCM implemented on Starmon-5 using readout correction.}
  \label{fig:pc_corrected_starmon_sphere}
\end{figure}

\begin{figure}[H]
  \centering
  \begin{subfigure}{.5\textwidth}
    \centering
    \includegraphics[width=\textwidth]{Figures/PhaseCovariant/IBM/FullSphere/results_corrected_athens_copy1.png}
    \label{fig:pc_corrected_athens_sphere_1}
  \end{subfigure}%
  \begin{subfigure}{.5\textwidth}
    \centering
    \includegraphics[width=\textwidth]{Figures/PhaseCovariant/IBM/FullSphere/results_corrected_athens_copy2.png}
    \label{fig:pc_corrected_athens_sphere_2}
  \end{subfigure}
  \caption{Fidelity sampled over the Bloch sphere using the PCQCM implemented on Athens using readout correction.}
  \label{fig:pc_corrected_athens_sphere}
\end{figure}


\begin{figure}[H]
  \centering
  \begin{subfigure}{.5\textwidth}
    \centering
    \includegraphics[width=\textwidth]{Figures/PhaseCovariant/IBM/FullSphere/results_corrected_ourense_copy1.png}
    
    \label{fig:pc_corrected_ourense_sphere_1}
  \end{subfigure}%
  \begin{subfigure}{.5\textwidth}
    \centering
    \includegraphics[width=\textwidth]{Figures/PhaseCovariant/IBM/FullSphere/results_corrected_ourense_copy2.png}
    
    \label{fig:pc_corrected_ourense_sphere_2}
  \end{subfigure}
  \caption{Fidelity sampled over the Bloch sphere using the PCQCM implemented on Ourense using readout correction.}
  \label{fig:pc_corrected_ourense_sphere}
\end{figure}

\begin{figure}[H]
  \centering
  \begin{subfigure}{.5\textwidth}
    \centering
    \includegraphics[width=\textwidth]{Figures/PhaseCovariant/IBM/FullSphere/results_corrected_santiago_copy1.png}
    
    \label{fig:pc_corrected_santiago_sphere_1}
  \end{subfigure}%
  \begin{subfigure}{.5\textwidth}
    \centering
    \includegraphics[width=\textwidth]{Figures/PhaseCovariant/IBM/FullSphere/results_corrected_santiago_copy2.png}
    
    \label{fig:pc_corrected_santiago_sphere_2}
  \end{subfigure}
  \caption{Fidelity sampled over the Bloch sphere using the PCQCM implemented on Santiago using readout correction.}
  \label{fig:pc_corrected_santiago_sphere}
\end{figure}

\begin{figure}[H]
  \centering
  \begin{subfigure}{.5\textwidth}
    \centering
    \includegraphics[width=\textwidth]{Figures/PhaseCovariant/IBM/FullSphere/results_corrected_valencia_copy1.png}
    \label{fig:pc_corrected_valencia_sphere_1}
  \end{subfigure}%
  \begin{subfigure}{.5\textwidth}
    \centering
    \includegraphics[width=\textwidth]{Figures/PhaseCovariant/IBM/FullSphere/results_corrected_valencia_copy2.png}
    \label{fig:pc_corrected_valencia_sphere_2}
  \end{subfigure}
  \caption{Fidelity sampled over the Bloch sphere using the PCQCM implemented on Valencia using readout correction.}
  \label{fig:pc_corrected_valencia_sphere}
\end{figure}


\begin{figure}[H]
  \centering
  \begin{subfigure}{.5\textwidth}
    \centering
    \includegraphics[width=\textwidth]{Figures/PhaseCovariant/IBM/FullSphere/results_corrected_vigo_copy1.png}
    
    \label{fig:pc_corrected_vigo_sphere_1}
  \end{subfigure}%
  \begin{subfigure}{.5\textwidth}
    \centering
    \includegraphics[width=\textwidth]{Figures/PhaseCovariant/IBM/FullSphere/results_corrected_vigo_copy2.png}
    
    \label{fig:pc_corrected_vigo_sphere_2}
  \end{subfigure}
  \caption{Fidelity sampled over the Bloch sphere using the PCQCM implemented on Vigo using readout correction.}
  \label{fig:pc_corrected_vigo_sphere}
\end{figure}

\begin{figure}[H]
  \centering
  \begin{subfigure}{.5\textwidth}
    \centering
    \includegraphics[width=\textwidth]{Figures/PhaseCovariant/IBM/FullSphere/results_corrected_ibmqx2_copy1.png}
    
    \label{fig:pc_corrected_yorktown_sphere_1}
  \end{subfigure}%
  \begin{subfigure}{.5\textwidth}
    \centering
    \includegraphics[width=\textwidth]{Figures/PhaseCovariant/IBM/FullSphere/results_corrected_ibmqx2_copy2.png}
    
    \label{fig:pc_corrected_yorktown_sphere_2}
  \end{subfigure}
  \caption{Fidelity sampled over the Bloch sphere using the PCQCM implemented on Yorktown using readout correction.}
  \label{fig:pc_corrected_yorktown_sphere}
\end{figure}

We conclude the analysis for the full sphere reporting also the results without readout correction.
\begin{table}[H]
    \centering
    \begin{tabular}{|c|c|c|c|c|c|c|c|}
    \hline
    \textbf{} & \textbf{Starmon-5} & \textbf{Athens} & \textbf{Ourense} & \textbf{Santiago} & \textbf{Valencia} & \textbf{Vigo} & \textbf{Yorktown} \\ \hline
    $F_1$              & 0.761 & 0.782 & 0.801 & 0.788 & 0.768 & 0.784 & 0.794 \\ \hline
    $\sigma_{F_1}$     & 0.036 & 0.035 & 0.029 & 0.032 & 0.044 & 0.030 & 0.023 \\ \hline
    $F_2$              & 0.757 & 0.825 & 0.779 & 0.772 & 0.797 & 0.807 & 0.793 \\ \hline
    $\sigma_{F_2}$     & 0.038 & 0.030 & 0.033 & 0.043 & 0.030 & 0.035 & 0.009 \\ \hline
    \end{tabular}
    \caption{Average fidelity and corresponding standard deviation on the whole Bloch sphere for the two copies on each backend. The results are not corrected with readout calibration.}
    \label{tab:results_pcqcm_fullsphere_not_corrected}
\end{table}
\begin{figure}[H]
  \centering
          \includegraphics[width=0.8\textwidth]{Figures/PhaseCovariant/Histograms/histo_sphere.png}
      \label{fig:pc_histo_sphere_not_corrected}
      \caption{Comparison of the average fidelity on the Bloch sphere for both the copies on different backends without using the PCQCM without readout correction. The uncertainty on the average is calculated as $\sigma_{\overline{F}_i}=\sigma_{F_i}/\sqrt{N}$, where $N$ is the number of points sampled ($N=1000$ in this case).}
\end{figure}
As discussed in the previous section for the UQCM, the readout calibration corrects a classical error associated to a measurement.
Therefore, we expect the measured fidelity to be more accurate, but not necessarily closer to the ideal case. 
Indeed, sometimes the readout correction increases the fidelity and sometimes it decreases it.

\subsection{Equator}
The fidelity for the copies of equatorial states using an ideal PCQCM is (see Equation \ref{eqn:fidelity_pcqcm}):
\[
  F=\frac{1}{2}+\frac{1}{2\sqrt{2}}\approx0.854.
\]
All the states should be copied equally well. However, we still expect a dispersion of the results because of the finite sampling (we are using $16384$ shots for each input state for Starmon-5 and $8192$ for the IBM backends).
As discussed previously in Chapter \ref{sec:implementation}, the expected statistical uncertainty is (see \ref{eqn:dispersion_pcqcm}):
\[
    \sigma_{stat}^{(8192)}\approx 0.004, \quad \sigma_{stat}^{(16384)}\approx 0.003.
\]
If the dispersion of the results is significantly greater than the expected one, it means that the states are not copied equally well.

The results we found are summarized in Table \ref{tab:results_pcqcm_onlyequator_corrected} (corrected with readout calibration).
\begin{table}[H]
    \centering
    \begin{tabular}{|c|c|c|c|c|c|c|c|}
    \hline
    \textbf{} & \textbf{Starmon-5} & \textbf{Athens} & \textbf{Ourense} & \textbf{Santiago} & \textbf{Valencia} & \textbf{Vigo} & \textbf{Yorktown} \\ \hline
    $F_1$              & 0.817 & 0.814 & 0.839 & 0.825 & 0.834 & 0.836 & 0.804 \\ \hline
    $\sigma_{F_1}$     & 0.014 & 0.019 & 0.018 & 0.014 & 0.023 & 0.022 & 0.030 \\ \hline
    $F_2$              & 0.815 & 0.858 & 0.811 & 0.857 & 0.836 & 0.839 & 0.819 \\ \hline
    $\sigma_{F_2}$     & 0.027 & 0.015 & 0.015 & 0.018 & 0.019 & 0.022 & 0.034 \\ \hline
    \end{tabular}
    \caption{Average fidelity and corresponding standard deviation on the $xz$-equator for the two copies on each backend. The results are corrected with readout calibration.}
    \label{tab:results_pcqcm_onlyequator_corrected}
\end{table}
\begin{figure}[H]
  \centering
          \includegraphics[width=0.8\textwidth]{Figures/PhaseCovariant/Histograms/histo_equator_corrected.png}
      \label{fig:pc_histo_equator_corrected}
      \caption{Comparison of the average fidelity on the $xz$-equator for both the copies on different backends using readout correction. The uncertainty on the average is calculated as $\sigma_{\overline{F}_i}=\sigma_{F_i}/\sqrt{N}$, where $N$ is the number of points sampled ($N=100$ in this case).}
\end{figure}
The average fidelity has significantly increased compared to the sampling on the whole sphere (and of course compared to the UQCM).
In several cases the fidelity is greater than the bound of the UQCM (i.e. $F\approx 0.833$). 
The average fidelity of the second copy of Athens and Santiago achieves the theoretical bound for the PCQCM (i.e. $F\approx 0.854$).

However, the measured dispersion of the results is significantly greater than the statistical uncertainty in every backend. 
This means that we can not conclude that all the input states on the $xz$-equator are copied equally well. 
This can be more clearly seen in the pictures below, where we have reported the measured fidelity of each input state.

There is a recurring pattern occuring in several backends, for example Athens (see Figure \ref{fig:pc_corrected_athens_equator}):
moving across the equator, the fidelity oscillates. There is a peak around $\ket{0}$ and $\ket{1}$ and a dip around $\ket{+}$ and $\ket{-}$.
For Vigo (see \ref{fig:pc_corrected_ourense_equator}) the oscillating pattern is not symmetric (the dips and the peaks are not at the same height). 
For Valencia (see \ref{fig:pc_corrected_valencia_equator}) the oscillating pattern is translated: the peaks are around $\ket{+}$ and $\ket{-}$, whereas the dips are around $\ket{0}$ and $\ket{1}$.
In all these cases, the patterns are consistent with the results obtained on $xz$-equator while sampling the full sphere. 
Considering, for example, the results for the second copy of Valencia in the right panel of Figure \ref{fig:pc_corrected_valencia_sphere}, it is possible to see a bright spot around $\ket{+}$ (exactly where we see the peak in Figure \ref{fig:pc_corrected_valencia_equator}).
For Santiago, we can again see the artefacts previously discussed. In this case they affect the last $25$ points, which are significantly displaced with respect to the first $75$.
Finally, we observe that the results for the two copies of Yorktown are qualitatively different, as previously seen: the fidelity of the first copy has a clear dependence from the angle, whereas the second one is uniformly distributed around the average value.

\begin{figure}[H]
  \centering
  \begin{subfigure}{.5\textwidth}
    \centering
    \includegraphics[width=\textwidth]{Figures/PhaseCovariant/Starmon/OnlyEquator/results_starmon_corrected_copy1.png}
    \label{fig:pc_corrected_starmon_equator_1}
  \end{subfigure}%
  \begin{subfigure}{.5\textwidth}
    \centering
    \includegraphics[width=\textwidth]{Figures/PhaseCovariant/Starmon/OnlyEquator/results_starmon_corrected_copy2.png}
    \label{fig:pc_corrected_starmon_equator_2}
  \end{subfigure}
  \vspace{-0.5cm}
  \caption{Fidelity sampled over the $xz$-equator using the PCQCM implemented on Starmon-5 using readout correction.
  The solid green line is the expected fidelity, the solid blue line is the measured average fidelity, the dashed red lines indicate the expected dispersion and the dashed blue line indicate the measured dispersion. The orange points are the $BB84$ states.}
  \label{fig:pc_corrected_starmon_equator}
\end{figure}

\begin{figure}[H]
  \centering
  \begin{subfigure}{.5\textwidth}
    \centering
    \includegraphics[width=\textwidth]{Figures/PhaseCovariant/IBM/OnlyEquator/results_corrected_athens_copy1.png}
    \label{fig:pc_corrected_athens_equator_1}
  \end{subfigure}%
  \begin{subfigure}{.5\textwidth}
    \centering
    \includegraphics[width=\textwidth]{Figures/PhaseCovariant/IBM/OnlyEquator/results_corrected_athens_copy2.png}
    \label{fig:pc_corrected_athens_equator_2}
  \end{subfigure}
  \vspace{-0.5cm}
  \caption{Fidelity sampled over the $xz$-equator using the PCQCM implemented on Athens using readout correction. The solid green line is the expected fidelity, the solid blue line is the measured average fidelity, the dashed red lines indicate the expected dispersion and the dashed blue line indicate the measured dispersion.}
  \label{fig:pc_corrected_athens_equator}
\end{figure}


\begin{figure}[H]
  \centering
  \begin{subfigure}{.5\textwidth}
    \centering
    \includegraphics[width=\textwidth]{Figures/PhaseCovariant/IBM/OnlyEquator/results_corrected_ourense_copy1.png}
    \label{fig:pc_corrected_ourense_equator_1}
  \end{subfigure}%
  \begin{subfigure}{.5\textwidth}
    \centering
    \includegraphics[width=\textwidth]{Figures/PhaseCovariant/IBM/OnlyEquator/results_corrected_ourense_copy2.png}
    \label{fig:pc_corrected_ourense_equator_2}
  \end{subfigure}
  \vspace{-0.5cm}
  \caption{Fidelity sampled over the $xz$-equator using the PCQCM implemented on Ourense using readout correction. The solid green line is the expected fidelity, the solid blue line is the measured average fidelity, the dashed red lines indicate the expected dispersion and the dashed blue line indicate the measured dispersion.}
  \label{fig:pc_corrected_ourense_equator}
\end{figure}


\begin{figure}[H]
  \centering
  \begin{subfigure}{.5\textwidth}
    \centering
    \includegraphics[width=\textwidth]{Figures/PhaseCovariant/IBM/OnlyEquator/results_corrected_santiago_copy1.png}
    \label{fig:pc_corrected_santiago_equator_1}
  \end{subfigure}%
  \begin{subfigure}{.5\textwidth}
    \centering
    \includegraphics[width=\textwidth]{Figures/PhaseCovariant/IBM/OnlyEquator/results_corrected_santiago_copy2.png}
    \label{fig:pc_corrected_santiago_equator_2}
  \end{subfigure}
  \vspace{-0.5cm}
  \caption{Fidelity sampled over the $xz$-equator using the PCQCM implemented on Santiago using readout correction. The solid green line is the expected fidelity, the solid blue line is the measured average fidelity, the dashed red lines indicate the expected dispersion and the dashed blue line indicate the measured dispersion.}
  \label{fig:pc_corrected_santiago_equator}
\end{figure}


\begin{figure}[H]
  \centering
  \begin{subfigure}{.5\textwidth}
    \centering
    \includegraphics[width=\textwidth]{Figures/PhaseCovariant/IBM/OnlyEquator/results_corrected_valencia_copy1.png}
    \label{fig:pc_corrected_valencia_equator_1}
  \end{subfigure}%
  \begin{subfigure}{.5\textwidth}
    \centering
    \includegraphics[width=\textwidth]{Figures/PhaseCovariant/IBM/OnlyEquator/results_corrected_valencia_copy2.png}
    \label{fig:pc_corrected_valencia_equator_2}
  \end{subfigure}
  \vspace{-0.5cm}
  \caption{Fidelity sampled over the $xz$-equator using the PCQCM implemented on Valencia using readout correction. The solid green line is the expected fidelity, the solid blue line is the measured average fidelity, the dashed red lines indicate the expected dispersion and the dashed blue line indicate the measured dispersion.}
  \label{fig:pc_corrected_valencia_equator}
\end{figure}


\begin{figure}[H]
  \centering
  \begin{subfigure}{.5\textwidth}
    \centering
    \includegraphics[width=\textwidth]{Figures/PhaseCovariant/IBM/OnlyEquator/results_corrected_vigo_copy1.png}
    \label{fig:pc_corrected_vigo_equator_1}
  \end{subfigure}%
  \begin{subfigure}{.5\textwidth}
    \centering
    \includegraphics[width=\textwidth]{Figures/PhaseCovariant/IBM/OnlyEquator/results_corrected_vigo_copy2.png}
    \label{fig:pc_corrected_vigo_equator_2}
  \end{subfigure}
  \vspace{-0.5cm}
  \caption{Fidelity sampled over the $xz$-equator using the PCQCM implemented on Vigo using readout correction. The solid green line is the expected fidelity, the solid blue line is the measured average fidelity, the dashed red lines indicate the expected dispersion and the dashed blue line indicate the measured dispersion.}
  \label{fig:pc_corrected_vigo_equator}
\end{figure}


\begin{figure}[H]
  \centering
  \begin{subfigure}{.5\textwidth}
    \centering
    \includegraphics[width=\textwidth]{Figures/PhaseCovariant/IBM/OnlyEquator/results_corrected_ibmqx2_copy1.png}
    \label{fig:pc_corrected_yorktown_equator_1}
  \end{subfigure}%
  \begin{subfigure}{.5\textwidth}
    \centering
    \includegraphics[width=\textwidth]{Figures/PhaseCovariant/IBM/OnlyEquator/results_corrected_ibmqx2_copy2.png}
    \label{fig:pc_corrected_yorktown_equator_2}
  \end{subfigure}
  \vspace{-0.5cm}
  \caption{Fidelity sampled over the $xz$-equator using the PCQCM implemented on Yorktown using readout correction. The solid green line is the expected fidelity, the solid blue line is the measured average fidelity, the dashed red lines indicate the expected dispersion and the dashed blue line indicate the measured dispersion.}
  \label{fig:pc_corrected_yorktown_equator}
\end{figure}


For the sake of completeness, we also report the non-corrected results.
\begin{table}[H]
  \centering
  \begin{tabular}{|c|c|c|c|c|c|c|c|}
  \hline
  \textbf{} & \textbf{Starmon-5} & \textbf{Athens} & \textbf{Ourense} & \textbf{Santiago} & \textbf{Valencia} & \textbf{Vigo} & \textbf{Yorktown} \\ \hline
  $F_1$              & 0.807 & 0.817 & 0.836 & 0.822 & 0.820 & 0.798 & 0.802 \\ \hline
  $\sigma_{F_1}$     & 0.014 & 0.019 & 0.017 & 0.013 & 0.017 & 0.019 & 0.028 \\ \hline
  $F_2$              & 0.798 & 0.858 & 0.811 & 0.815 & 0.836 & 0.836 & 0.807 \\ \hline
  $\sigma_{F_2}$     & 0.025 & 0.015 & 0.015 & 0.011 & 0.018 & 0.021 & 0.008 \\ \hline
  \end{tabular}
  \caption{Average fidelity and corresponding standard deviation on the $xz$-equator for the two copies on each backend. The results are not corrected with readout calibration.}
  \label{tab:results_pcqcm_onlyequator_not_corrected}
\end{table}
\begin{figure}[H]
  \centering
          \includegraphics[width=0.8\textwidth]{Figures/PhaseCovariant/Histograms/histo_equator.png}
      \label{fig:pc_histo_equator_not_corrected}
      \caption{Comparison of the average fidelity on the $xz$-equator for both the copies on different backends without using the PCQCM without readout correction. The uncertainty on the average is calculated as $\sigma_{\overline{F}_i}=\sigma_{F_i}/\sqrt{N}$, where $N$ is the number of points sampled ($N=100$ in this case).}
\end{figure}

\subsection{BB84 states}
In Table \ref{tab:results_pcqcm_bb84_corrected} are presented the results obtained considering the BB84 states as input states, considering the readout correction.
There is not a significant difference when comparing these results with the one obtained for the full equator.
This is reasonable: since the BB84 states correspond to the peaks or dips of the fidelity on the equator discussed before, 
both the average and the standard deviation should remain more or less unchanged.

\begin{table}[H]
  \centering
  \begin{tabular}{|c|c|c|c|c|c|c|c|}
  \hline
  \textbf{} & \textbf{Starmon-5} & \textbf{Athens} & \textbf{Ourense} & \textbf{Santiago} & \textbf{Valencia} & \textbf{Vigo} & \textbf{Yorktown} \\ \hline
  $F_1$              & 0.838 & 0.818 & 0.840 & 0.833 & 0.841 & 0.820 & 0.807 \\ \hline
  $\sigma_{F_1}$     & 0.027 & 0.023 & 0.022 & 0.011 & 0.009 & 0.031 & 0.028 \\ \hline
  $F_2$              & 0.808 & 0.852 & 0.817 & 0.833 & 0.820 & 0.830 & 0.788 \\ \hline
  $\sigma_{F_2}$     & 0.029 & 0.024 & 0.018 & 0.016 & 0.005 & 0.032 & 0.038 \\ \hline
  \end{tabular}
  \caption{Average fidelity and corresponding standard deviation for the BB84 for the two copies on each backend. The results are corrected with readout calibration.}
  \label{tab:results_pcqcm_bb84_corrected}
\end{table}
\begin{figure}[H]
  \centering
          \includegraphics[width=0.8\textwidth]{Figures/PhaseCovariant/Histograms/histo_bb84_corrected.png}
      \label{fig:pc_histo_bb84_corrected}
      \caption{Comparison of the average fidelity for the BB84 states for both the copies on different backends using the PCQCM with readout correction. The uncertainty on the average is calculated as $\sigma_{\overline{F}_i}=\sigma_{F_i}/\sqrt{N}$, where $N$ is the number of points sampled ($N=4$ in this case).}
\end{figure}

We conclude the discusison of the results for the PCQCM with the results for the BB84 states without readout correction.
\begin{table}[H]
    \centering
    \begin{tabular}{|c|c|c|c|c|c|c|c|}
    \hline
    \textbf{} & \textbf{Starmon-5} & \textbf{Athens} & \textbf{Ourense} & \textbf{Santiago} & \textbf{Valencia} & \textbf{Vigo} & \textbf{Yorktown} \\ \hline
    $F_1$              & 0.802 & 0.820 & 0.838 & 0.829 & 0.821 & 0.797 & 0.806 \\ \hline
    $\sigma_{F_1}$     & 0.024 & 0.023 & 0.021 & 0.011 & 0.008 & 0.026 & 0.027 \\ \hline
    $F_2$              & 0.796 & 0.853 & 0.815 & 0.832 & 0.824 & 0.829 & 0.793 \\ \hline
    $\sigma_{F_2}$     & 0.027 & 0.023 & 0.017 & 0.015 & 0.005 & 0.030 & 0.008 \\ \hline
    \end{tabular}
    \caption{Average fidelity and corresponding standard deviation for the BB84 for the two copies on each backend. The results are not corrected with readout calibration.}
    \label{tab:results_pcqcm_bb84_not_corrected}
\end{table}
\begin{figure}[H]
  \centering
          \includegraphics[width=0.8\textwidth]{Figures/PhaseCovariant/Histograms/histo_bb84.png}
      \label{fig:pc_histo_bb84_corrected}
      \caption{Comparison of the average fidelity for the BB84 states for both the copies on different backends using the PCQCM without readout correction. The uncertainty on the average is calculated as $\sigma_{\overline{F}_i}=\sigma_{F_i}/\sqrt{N}$, where $N$ is the number of points sampled ($N=4$ in this case).}
\end{figure}

\section{Economical phase covariant quantum cloning machine}
\subsection{Sphere}

Things to mention:
\begin{enumerate}
  \item Pattern of ideal case well reproduced. Yorktown has some differences again, as seen in the previous sections too.
  \item Readout correction influences greatly the result of Spin-2. 
  \item Having fidelity greater than the ideal one should not be a concern. Half the Bloch sphere is copied better than the equator. 
  It could be that the prepartion and measurement of the points is not very precise. For example, it could be that instead of preparing and measuring the state $\ket{+}$, we are preparing and measuring a slightly rotated version which in the better emisphere.
\end{enumerate} 

With readout correction:
\begin{table}[H]
    \centering
    \begin{tabular}{|c|c|c|c|c|c|c|c|c|}
    \hline
    \textbf{} & \textbf{Spin-2} & \textbf{Starmon-5} & \textbf{Athens} & \textbf{Ourense} & \textbf{Santiago} & \textbf{Valencia} & \textbf{Vigo} & \textbf{Yorktown} \\ \hline
    $F_1$              & 0.731 & 0.753 & 0.810 & 0.777 & 0.814 & 0.820 & 0.824 & 0.781 \\ \hline
    $\sigma_{F_1}$     & 0.172 & 0.151 & 0.144 & 0.166 & 0.138 & 0.138 & 0.131 & 0.140 \\ \hline
    $F_2$              & 0.821 & 0.791 & 0.811 & 0.844 & 0.799 & 0.807 & 0.801 & 0.776 \\ \hline
    $\sigma_{F_2}$     & 0.130 & 0.133 & 0.139 & 0.127 & 0.153 & 0.153 & 0.154 & 0.145 \\ \hline
    \end{tabular}
\end{table}
\begin{figure}[H]
  \centering
          \includegraphics[width=0.8\textwidth]{Figures/Economical/Histograms/histo_sphere_corrected.png}
      \label{fig:epc_histo_sphere_corrected}
      \caption{Comparison of the average fidelity on the Bloch sphere for both the copies on different backends using the EPCQCM with readout correction. The uncertainty on the average is calculated as $\sigma_{\overline{F}_i}=\sigma_{F_i}/\sqrt{N}$, where $N$ is the number of points sampled ($N=1000$ in this case).}
\end{figure}

\begin{figure}[H]
  \centering
  \begin{subfigure}{.5\textwidth}
    \centering
    \includegraphics[width=\textwidth]{Figures/Economical/Spin2/FullSphere/results_spin2_corrected_copy1.png}
    \label{fig:epc_corrected_spin_sphere_1}
  \end{subfigure}%
  \begin{subfigure}{.5\textwidth}
    \centering
    \includegraphics[width=\textwidth]{Figures/Economical/Spin2/FullSphere/results_spin2_corrected_copy2.png}
    \label{fig:epc_corrected_spin_sphere_2}
  \end{subfigure}
  \caption{Fidelity sampled over the Bloch sphere using the EPCQCM implemented on Spin-2 using readout correction.}
  \label{fig:epc_corrected_spin_sphere}
\end{figure}

\begin{figure}[H]
  \centering
  \begin{subfigure}{.5\textwidth}
    \centering
    \includegraphics[width=\textwidth]{Figures/Economical/Starmon/FullSphere/results_starmon_corrected_copy1.png}
    \label{fig:epc_corrected_starmon_sphere_1}
  \end{subfigure}%
  \begin{subfigure}{.5\textwidth}
    \centering
    \includegraphics[width=\textwidth]{Figures/Economical/Starmon/FullSphere/results_starmon_corrected_copy2.png}
    \label{fig:epc_corrected_starmon_sphere_2}
  \end{subfigure}
  \caption{Fidelity sampled over the Bloch sphere using the EPCQCM implemented on Starmon-5 using readout correction.}
  \label{fig:epc_corrected_starmon_sphere}
\end{figure}

\begin{figure}[H]
  \centering
  \begin{subfigure}{.5\textwidth}
    \centering
    \includegraphics[width=\textwidth]{Figures/Economical/IBM/FullSphere/results_corrected_athens_copy1.png}
    \label{fig:epc_corrected_athens_sphere_1}
  \end{subfigure}%
  \begin{subfigure}{.5\textwidth}
    \centering
    \includegraphics[width=\textwidth]{Figures/Economical/IBM/FullSphere/results_corrected_athens_copy2.png}
    \label{fig:epc_corrected_athens_sphere_2}
  \end{subfigure}
  \caption{Fidelity sampled over the Bloch sphere using the EPCQCM implemented on Athens using readout correction.}
  \label{fig:epc_corrected_athens_sphere}
\end{figure}

\begin{figure}[H]
  \centering
  \begin{subfigure}{.5\textwidth}
    \centering
    \includegraphics[width=\textwidth]{Figures/Economical/IBM/FullSphere/results_corrected_ourense_copy1.png}
    
    \label{fig:epc_corrected_ourense_sphere_1}
  \end{subfigure}%
  \begin{subfigure}{.5\textwidth}
    \centering
    \includegraphics[width=\textwidth]{Figures/Economical/IBM/FullSphere/results_corrected_ourense_copy2.png}
    
    \label{fig:epc_corrected_ourense_sphere_2}
  \end{subfigure}
  \caption{Fidelity sampled over the Bloch sphere using the EPCQCM implemented on Ourense using readout correction.}
  \label{fig:epc_corrected_ourense_sphere}
\end{figure}

\begin{figure}[H]
  \centering
  \begin{subfigure}{.5\textwidth}
    \centering
    \includegraphics[width=\textwidth]{Figures/Economical/IBM/FullSphere/results_corrected_santiago_copy1.png}
    
    \label{fig:epc_corrected_santiago_sphere_1}
  \end{subfigure}%
  \begin{subfigure}{.5\textwidth}
    \centering
    \includegraphics[width=\textwidth]{Figures/Economical/IBM/FullSphere/results_corrected_santiago_copy2.png}
    
    \label{fig:epc_corrected_santiago_sphere_2}
  \end{subfigure}
  \caption{Fidelity sampled over the Bloch sphere using the EPCQCM implemented on Santiago using readout correction.}
  \label{fig:epc_corrected_santiago_sphere}
\end{figure}

\begin{figure}[H]
  \centering
  \begin{subfigure}{.5\textwidth}
    \centering
    \includegraphics[width=\textwidth]{Figures/Economical/IBM/FullSphere/results_corrected_valencia_copy1.png}
    \label{fig:epc_corrected_valencia_sphere_1}
  \end{subfigure}%
  \begin{subfigure}{.5\textwidth}
    \centering
    \includegraphics[width=\textwidth]{Figures/Economical/IBM/FullSphere/results_corrected_valencia_copy2.png}
    \label{fig:epc_corrected_valencia_sphere_2}
  \end{subfigure}
  \caption{Fidelity sampled over the Bloch sphere using the EPCQCM implemented on Valencia using readout correction.}
  \label{fig:epc_corrected_valencia_sphere}
\end{figure}

\begin{figure}[H]
  \centering
  \begin{subfigure}{.5\textwidth}
    \centering
    \includegraphics[width=\textwidth]{Figures/Economical/IBM/FullSphere/results_corrected_vigo_copy1.png}
    
    \label{fig:epc_corrected_vigo_sphere_1}
  \end{subfigure}%
  \begin{subfigure}{.5\textwidth}
    \centering
    \includegraphics[width=\textwidth]{Figures/Economical/IBM/FullSphere/results_corrected_vigo_copy2.png}
    
    \label{fig:epc_corrected_vigo_sphere_2}
  \end{subfigure}
  \caption{Fidelity sampled over the Bloch sphere using the EPCQCM implemented on Vigo using readout correction.}
  \label{fig:epc_corrected_vigo_sphere}
\end{figure}

\begin{figure}[H]
  \centering
  \begin{subfigure}{.5\textwidth}
    \centering
    \includegraphics[width=\textwidth]{Figures/Economical/IBM/FullSphere/results_corrected_ibmqx2_copy1.png}
    
    \label{fig:epc_corrected_yorktown_sphere_1}
  \end{subfigure}%
  \begin{subfigure}{.5\textwidth}
    \centering
    \includegraphics[width=\textwidth]{Figures/Economical/IBM/FullSphere/results_corrected_ibmqx2_copy2.png}
    
    \label{fig:epc_corrected_yorktown_sphere_2}
  \end{subfigure}
  \caption{Fidelity sampled over the Bloch sphere using the EPCQCM implemented on Yorktown using readout correction.}
  \label{fig:epc_corrected_yorktown_sphere}
\end{figure}

Without readout correction:
\begin{table}[H]
    \centering
    \begin{tabular}{|c|c|c|c|c|c|c|c|c|}
    \hline
    \textbf{} & \textbf{Spin-2} & \textbf{Starmon-5} & \textbf{Athens} & \textbf{Ourense} & \textbf{Santiago} & \textbf{Valencia} & \textbf{Vigo} & \textbf{Yorktown} \\ \hline
    $F_1$              & 0.714  & 0.741 & 0.810 & 0.776 & 0.811 & 0.819 & 0.803 & 0.789 \\ \hline
    $\sigma_{F_1}$     & 0.128  & 0.142 & 0.142 & 0.159 & 0.132 & 0.132 & 0.112 & 0.129 \\ \hline
    $F_2$              & 0.690  & 0.788 & 0.812 & 0.834 & 0.794 & 0.826 & 0.802 & 0.777 \\ \hline
    $\sigma_{F_2}$     & 0.083  & 0.126 & 0.136 & 0.119 & 0.149 & 0.133 & 0.149 & 0.137 \\ \hline
    \end{tabular}
\end{table}
\begin{figure}[H]
  \centering
          \includegraphics[width=0.8\textwidth]{Figures/Economical/Histograms/histo_sphere.png}
      \label{fig:epc_histo_sphere_not_corrected}
      \caption{Comparison of the average fidelity on the Bloch sphere for both the copies on different backends without using the EPCQCM without readout correction. The uncertainty on the average is calculated as $\sigma_{\overline{F}_i}=\sigma_{F_i}/\sqrt{N}$, where $N$ is the number of points sampled ($N=1000$ in this case).}
\end{figure}

\subsection{Equator}
Things to mention:
\begin{enumerate}
  \item Oscillatory pattern again (with few exceptions). Also details of oscillatory pattern are similar (e.g. Vigo is not symmetric and Valencia is translated, so $\ket{+}$ and $\ket{-}$ are copied better)
  \item Readout correction influences greatly the result of Spin-2. 
  \item Copy 1 of Valencia is almost ideal (both for the fidelity and the dispersion)
  \item Having fidelity greater than the ideal one should not be a concern. Half the Bloch sphere is copied better than the equator. 
  It could be that the prepartion and measurement of the points is not very precise. For example, it could be that instead of preparing and measuring the state $\ket{+}$, we are preparing and measuring a slightly rotated version which in the better emisphere.
\end{enumerate} 

With readout correction:
\begin{table}[H]
    \centering
    \begin{tabular}{|c|c|c|c|c|c|c|c|c|}
    \hline
    \textbf{} & \textbf{Spin-2} & \textbf{Starmon-5} & \textbf{Athens} & \textbf{Ourense} & \textbf{Santiago} & \textbf{Valencia} & \textbf{Vigo} & \textbf{Yorktown} \\ \hline
    $F_1$              & 0.751 & 0.792 & 0.840 & 0.815 & 0.838 & 0.851 & 0.850 & 0.814 \\ \hline
    $\sigma_{F_1}$     & 0.042 & 0.022 & 0.011 & 0.019 & 0.010 & 0.005 & 0.014 & 0.041 \\ \hline
    $F_2$              & 0.862 & 0.846 & 0.837 & 0.879 & 0.811 & 0.832 & 0.828 & 0.835 \\ \hline
    $\sigma_{F_2}$     & 0.037 & 0.023 & 0.009 & 0.022 & 0.019 & 0.015 & 0.019 & 0.028 \\ \hline
    \end{tabular}
\end{table}
\begin{figure}[H]
  \centering
          \includegraphics[width=0.8\textwidth]{Figures/Economical/Histograms/histo_equator_corrected.png}
      \label{fig:epc_histo_equator_corrected}
      \caption{Comparison of the average fidelity on the $xz$-equator for both the copies on different backends using the EPCQCM with readout correction. The uncertainty on the average is calculated as $\sigma_{\overline{F}_i}=\sigma_{F_i}/\sqrt{N}$, where $N$ is the number of points sampled ($N=100$ in this case).}
\end{figure}

\begin{figure}[H]
  \centering
  \begin{subfigure}{.5\textwidth}
    \centering
    \includegraphics[width=\textwidth]{Figures/Economical/Spin2/OnlyEquator/results_spin2_corrected_copy1.png}
    \label{fig:epc_corrected_spin2_equator_1}
  \end{subfigure}%
  \begin{subfigure}{.5\textwidth}
    \centering
    \includegraphics[width=\textwidth]{Figures/Economical/Spin2/OnlyEquator/results_spin2_corrected_copy2.png}
    \label{fig:epc_corrected_spin2_equator_2}
  \end{subfigure}
  \vspace{-0.5cm}
  \caption{Fidelity sampled over the $xz$-equator using the EPCQCM implemented on Spin-2 using readout correction.
  The solid green line is the expected fidelity, the solid blue line is the measured average fidelity, the dashed red lines indicate the expected dispersion and the dashed blue line indicate the measured dispersion. The orange points are the $BB84$ states.}
  \label{fig:epc_corrected_spin2_equator}
\end{figure}

\begin{figure}[H]
  \centering
  \begin{subfigure}{.5\textwidth}
    \centering
    \includegraphics[width=\textwidth]{Figures/Economical/Starmon/OnlyEquator/results_starmon_corrected_copy1.png}
    \label{fig:epc_corrected_starmon_equator_1}
  \end{subfigure}%
  \begin{subfigure}{.5\textwidth}
    \centering
    \includegraphics[width=\textwidth]{Figures/Economical/Starmon/OnlyEquator/results_starmon_corrected_copy2.png}
    \label{fig:epc_corrected_starmon_equator_2}
  \end{subfigure}
  \vspace{-0.5cm}
  \caption{Fidelity sampled over the $xz$-equator using the EPCQCM implemented on Starmon-5 using readout correction.
  The solid green line is the expected fidelity, the solid blue line is the measured average fidelity, the dashed red lines indicate the expected dispersion and the dashed blue line indicate the measured dispersion. The orange points are the $BB84$ states.}
  \label{fig:epc_corrected_starmon_equator}
\end{figure}

\begin{figure}[H]
  \centering
  \begin{subfigure}{.5\textwidth}
    \centering
    \includegraphics[width=\textwidth]{Figures/Economical/IBM/OnlyEquator/results_corrected_athens_copy1.png}
    \label{fig:epc_corrected_athens_equator_1}
  \end{subfigure}%
  \begin{subfigure}{.5\textwidth}
    \centering
    \includegraphics[width=\textwidth]{Figures/Economical/IBM/OnlyEquator/results_corrected_athens_copy2.png}
    \label{fig:epc_corrected_athens_equator_2}
  \end{subfigure}
  \vspace{-0.5cm}
  \caption{Fidelity sampled over the $xz$-equator using the EPCQCM implemented on Athens using readout correction. The solid green line is the expected fidelity, the solid blue line is the measured average fidelity, the dashed red lines indicate the expected dispersion and the dashed blue line indicate the measured dispersion.}
  \label{fig:epc_corrected_athens_equator}
\end{figure}

\begin{figure}[H]
  \centering
  \begin{subfigure}{.5\textwidth}
    \centering
    \includegraphics[width=\textwidth]{Figures/Economical/IBM/OnlyEquator/results_corrected_ourense_copy1.png}
    \label{fig:epc_corrected_ourense_equator_1}
  \end{subfigure}%
  \begin{subfigure}{.5\textwidth}
    \centering
    \includegraphics[width=\textwidth]{Figures/Economical/IBM/OnlyEquator/results_corrected_ourense_copy2.png}
    \label{fig:epc_corrected_ourense_equator_2}
  \end{subfigure}
  \vspace{-0.5cm}
  \caption{Fidelity sampled over the $xz$-equator using the EPCQCM implemented on Ourense using readout correction. The solid green line is the expected fidelity, the solid blue line is the measured average fidelity, the dashed red lines indicate the expected dispersion and the dashed blue line indicate the measured dispersion.}
  \label{fig:epc_corrected_ourense_equator}
\end{figure}

\begin{figure}[H]
  \centering
  \begin{subfigure}{.5\textwidth}
    \centering
    \includegraphics[width=\textwidth]{Figures/Economical/IBM/OnlyEquator/results_corrected_santiago_copy1.png}
    \label{fig:epc_corrected_santiago_equator_1}
  \end{subfigure}%
  \begin{subfigure}{.5\textwidth}
    \centering
    \includegraphics[width=\textwidth]{Figures/Economical/IBM/OnlyEquator/results_corrected_santiago_copy2.png}
    \label{fig:epc_corrected_santiago_equator_2}
  \end{subfigure}
  \vspace{-0.5cm}
  \caption{Fidelity sampled over the $xz$-equator using the EPCQCM implemented on Santiago using readout correction. The solid green line is the expected fidelity, the solid blue line is the measured average fidelity, the dashed red lines indicate the expected dispersion and the dashed blue line indicate the measured dispersion.}
  \label{fig:epc_corrected_santiago_equator}
\end{figure}

\begin{figure}[H]
  \centering
  \begin{subfigure}{.5\textwidth}
    \centering
    \includegraphics[width=\textwidth]{Figures/Economical/IBM/OnlyEquator/results_corrected_valencia_copy1.png}
    \label{fig:epc_corrected_valencia_equator_1}
  \end{subfigure}%
  \begin{subfigure}{.5\textwidth}
    \centering
    \includegraphics[width=\textwidth]{Figures/Economical/IBM/OnlyEquator/results_corrected_valencia_copy2.png}
    \label{fig:epc_corrected_valencia_equator_2}
  \end{subfigure}
  \vspace{-0.5cm}
  \caption{Fidelity sampled over the $xz$-equator using the EPCQCM implemented on Valencia using readout correction. The solid green line is the expected fidelity, the solid blue line is the measured average fidelity, the dashed red lines indicate the expected dispersion and the dashed blue line indicate the measured dispersion.}
  \label{fig:epc_corrected_valencia_equator}
\end{figure}

\begin{figure}[H]
  \centering
  \begin{subfigure}{.5\textwidth}
    \centering
    \includegraphics[width=\textwidth]{Figures/Economical/IBM/OnlyEquator/results_corrected_vigo_copy1.png}
    \label{fig:epc_corrected_vigo_equator_1}
  \end{subfigure}%
  \begin{subfigure}{.5\textwidth}
    \centering
    \includegraphics[width=\textwidth]{Figures/Economical/IBM/OnlyEquator/results_corrected_vigo_copy2.png}
    \label{fig:epc_corrected_vigo_equator_2}
  \end{subfigure}
  \vspace{-0.5cm}
  \caption{Fidelity sampled over the $xz$-equator using the EPCQCM implemented on Vigo using readout correction. The solid green line is the expected fidelity, the solid blue line is the measured average fidelity, the dashed red lines indicate the expected dispersion and the dashed blue line indicate the measured dispersion.}
  \label{fig:epc_corrected_vigo_equator}
\end{figure}

\begin{figure}[H]
  \centering
  \begin{subfigure}{.5\textwidth}
    \centering
    \includegraphics[width=\textwidth]{Figures/Economical/IBM/OnlyEquator/results_corrected_ibmqx2_copy1.png}
    \label{fig:epc_corrected_yorktown_equator_1}
  \end{subfigure}%
  \begin{subfigure}{.5\textwidth}
    \centering
    \includegraphics[width=\textwidth]{Figures/Economical/IBM/OnlyEquator/results_corrected_ibmqx2_copy2.png}
    \label{fig:epc_corrected_yorktown_equator_2}
  \end{subfigure}
  \vspace{-0.5cm}
  \caption{Fidelity sampled over the $xz$-equator using the EPCQCM implemented on Yorktown using readout correction. The solid green line is the expected fidelity, the solid blue line is the measured average fidelity, the dashed red lines indicate the expected dispersion and the dashed blue line indicate the measured dispersion.}
  \label{fig:epc_corrected_yorktown_equator}
\end{figure}

Without readout correction:
\begin{table}[H]
    \centering
    \begin{tabular}{|c|c|c|c|c|c|c|c|c|}
    \hline
    \textbf{} & \textbf{Spin-2} & \textbf{Starmon-5} & \textbf{Athens} & \textbf{Ourense} & \textbf{Santiago} & \textbf{Valencia} & \textbf{Vigo} & \textbf{Yorktown} \\ \hline
    $F_1$              & 0.751  & 0.784 & 0.839 & 0.813 & 0.837 & 0.852 & 0.824 & 0.819 \\ \hline
    $\sigma_{F_1}$     & 0.029  & 0.021 & 0.010 & 0.018 & 0.009 & 0.005 & 0.012 & 0.037 \\ \hline
    $F_2$              & 0.704  & 0.841 & 0.838 & 0.867 & 0.799 & 0.850 & 0.826 & 0.832 \\ \hline
    $\sigma_{F_2}$     & 0.025  & 0.022 & 0.009 & 0.019 & 0.022 & 0.012 & 0.018 & 0.026 \\ \hline
    \end{tabular}
\end{table}
\begin{figure}[H]
  \centering
          \includegraphics[width=0.8\textwidth]{Figures/Economical/Histograms/histo_equator.png}
      \label{fig:epc_histo_equator_not_corrected}
      \caption{Comparison of the average fidelity on the $xz$-equator for both the copies on different backends without using the EPCQCM without readout correction. The uncertainty on the average is calculated as $\sigma_{\overline{F}_i}=\sigma_{F_i}/\sqrt{N}$, where $N$ is the number of points sampled ($N=100$ in this case).}
\end{figure}

\subsection{BB84 states}
Things to mention:
\begin{enumerate}
  \item Same considerations as for the equator hold.
\end{enumerate} 

With readout correction:
\begin{table}[H]
    \centering
    \begin{tabular}{|c|c|c|c|c|c|c|c|c|}
    \hline
    \textbf{} & \textbf{Spin-2} & \textbf{Starmon-5} & \textbf{Athens} & \textbf{Ourense} & \textbf{Santiago} & \textbf{Valencia} & \textbf{Vigo} & \textbf{Yorktown} \\ \hline
    $F_1$              & 0.823 & 0.801 & 0.841 & 0.815 & 0.851 & 0.858 & 0.850 & 0.809 \\ \hline
    $\sigma_{F_1}$     & 0.027 & 0.025 & 0.008 & 0.028 & 0.008 & 0.003 & 0.020 & 0.029 \\ \hline
    $F_2$              & 0.847 & 0.848 & 0.840 & 0.874 & 0.833 & 0.843 & 0.829 & 0.835 \\ \hline
    $\sigma_{F_2}$     & 0.041 & 0.024 & 0.003 & 0.021 & 0.012 & 0.011 & 0.023 & 0.023 \\ \hline
    \end{tabular}
\end{table}
\begin{figure}[H]
  \centering
          \includegraphics[width=0.8\textwidth]{Figures/Economical/Histograms/histo_bb84_corrected.png}
      \label{fig:epc_histo_bb84_corrected}
      \caption{Comparison of the average fidelity for the BB84 states for both the copies on different backends using the EPCQCM with readout correction. The uncertainty on the average is calculated as $\sigma_{\overline{F}_i}=\sigma_{F_i}/\sqrt{N}$, where $N$ is the number of points sampled ($N=4$ in this case).}
\end{figure}

Without readout correction:
\begin{table}[H]
    \centering
    \begin{tabular}{|c|c|c|c|c|c|c|c|c|}
    \hline
    \textbf{} & \textbf{Spin-2} & \textbf{Starmon-5} & \textbf{Athens} & \textbf{Ourense} & \textbf{Santiago} & \textbf{Valencia} & \textbf{Vigo} & \textbf{Yorktown} \\ \hline
    $F_1$              & 0.666  & 0.788 & 0.840 & 0.812 & 0.846 & 0.844 & 0.832 & 0.814 \\ \hline
    $\sigma_{F_1}$     & 0.018  & 0.024 & 0.008 & 0.027 & 0.008 & 0.003 & 0.017 & 0.027 \\ \hline
    $F_2$              & 0.765  & 0.845 & 0.842 & 0.866 & 0.831 & 0.848 & 0.827 & 0.833 \\ \hline
    $\sigma_{F_2}$     & 0.022  & 0.023 & 0.003 & 0.020 & 0.012 & 0.010 & 0.022 & 0.022 \\ \hline
    \end{tabular}
\end{table}
\begin{figure}[H]
  \centering
          \includegraphics[width=0.8\textwidth]{Figures/Economical/Histograms/histo_bb84.png}
      \label{fig:epc_histo_bb84_not_corrected}
      \caption{Comparison of the average fidelity for the BB84 states for both the copies on different backends using the EPCQCM without readout correction. The uncertainty on the average is calculated as $\sigma_{\overline{F}_i}=\sigma_{F_i}/\sqrt{N}$, where $N$ is the number of points sampled ($N=4$ in this case).}
\end{figure}